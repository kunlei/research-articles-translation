\section{Experiments}
In this paper, we randomly generated 50 arrays of size 10 with values no greater than 20.
All the arrays are sorted non-decreasingly and all the equal numbers are set to 0.
Put the 50 arrays into five groups: 1) the first group encompasses arrays 1 $\sim$ 10; 2) the second group encompasses arrays 1 $\sim$ 20; 3) the third group encompasses arrays 1 $\sim$ 30; 4) the first group encompasses arrays 1 $\sim$ 40; 5) the first group encompasses arrays 1 $\sim$ 50;
These generated instances are solved using the modified CCA (MCCA) and the minimal hitting set algorithm based on dynamic maximum (DMDSE-Tree).
For each group, the minimal hitting set and average computation times from five runs are recorded to make further comparisons.

The experiments are run on a Windows XP computer with Intel Pentium G630 2.70 GHz CPU and 2GB memory.
All the algorithms are implemented using Visual C++ 6.0.
The parameters of MCCA are set as follows: $N_{pop} = 100$, $N_{imp} = 7$, $N_{ind}  = 5$, the maximum number of iterations $N_{max} = 100$, the weight is set as $0.2, 0.3, \cdots, 0.8$ and $\alpha = 0.8$.
The minimal hitting set and computational time comparisons between MCCA and DMDSE-Tree are given table ? and table ?, respectively.








The experiments are run fives times and the average computation time and minimal hitting sets are obtained.
Table ? shows the comparisons between computation time percentage and number of minimal hitting sets.



It can be seen from table ? that MCCA is able to solve 90\% of instances with about 70\% of computational times when compared to DMDSE-Tree, which shows the superior performance of MCCA in solving minimal hitting set problem.
In addition, MCCA shows better robustness in solving problems with various scales.