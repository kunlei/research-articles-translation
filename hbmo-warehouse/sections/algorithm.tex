\section{Application in equipment selection}
\subsection{Algorithm workflow}
The warehouse facility model is subject to various constraints defined in the previous section, and exact methods like branch-and-bound and mathematical programming take very long time to reach optimal solutions.
On the other hand, metaheuristic algorithms have been successfully applied in many engineering optimization problems.
Bee colony optimization (BCO) is a metaheuristic algorithm that imitates the mating process of bee colonies.
A bee colony consists of four types of bees, namely, queen bee, drone, worker bee and bee larva. 
There usually exist only one queen bee within a colony and it is often the one who lives the longest.
The queen bee aims to mate with drones to produce larvae.
A drone is a haploid that mates with the queen bee and dies after the mating process.
Genes of the drone enter the queen bee and passe along to larvae.
Worker bee exists to produce honey to provide food for larvae, which grow up to be either queen bee or drones.
Abbass proposed a bee colony optimization algorithm based on the observation of a bee colony's mating and reproduce process, and successfully applied it to various engineering optimization problems.


In BCO, bees, including the queen bee, drones and larvae, represent individual solutions to the problem at hand and all the constraints are embedded in the solution representation.
In other words, a bee represents a feasible solution to the warehouse facility location problem depicted by the mathematical model described in the previous section.
In BCO, the number of times that the queen bee mates is determined by the number of drones, the results of the mating process decide the size of larvae.
The size of the queen bee's ovary indicates the storage size of drones' genes.
The number of iterations indicates the number of mating process.
The workflow of BCO can be summarised as follows:
\begin{itemize}
	\item Initialize parameters of the BCO.
	\item Create the initial population and calculate their fitness values. Selected the bee with the best fitness value and use it as the queen bee, the rest bees are used as drones.
	\item Check whether stopping criteria is met, stop if so, go to the next step otherwise.
	\item Initialize the ovary, energy and speed of the queen bee. 
	Repeat the following steps if the energy is bigger than the threshold and ovary is not full: 1) choose a drone and let it mate with the queen been if mating criteria is met, put its genes into the ovary; 2) decrease the energy and speed of the queen bee.
	\item The queen bee selects sequentially the genes in its ovary to produce larvae, which is then raised by different worker bees. If the resulting bee has a better fitness value than the queen bee, it becomes the new queen bee. Similarly it becomes a drone if its fitness value is better than any of the drones.
	\item Output the queen bee.
\end{itemize}



\subsection{Encoding scheme}


\subsection{Fitness function}


\subsection{Crossover operator}




\subsection{Local search algorithm}




