\section{Introduction}
In rail transit network systems, warehouse facilities play a key role in delivering  the required equipment and materials in a accurate and timely manner, based on the needs of each lane.
Warehouse facility locations therefore greatly affect whether a rail transit network system can operate efficiently at minimal costs.
Moreover, the decision has a long-lasting effect on  operation costs in future years as these facilities are hard to change once they are determined.
An optimal warehouse facility location solution is desired in order to satisfy the requirements of each lane and minimise the total operation costs.
It is common in current practices to set up a dedicated warehouse for each rail lane, which not only consumes more lands and investment, but also leads to duplicated and wasted stock of equipment and materials.
With the advance of rail transit systems in China, it is anticipated that they will be built in more cities, with major ones witnessing as many as seven or eight rail lanes.
It remains an urgent task to identify the best warehouse facility locations for these systems.


