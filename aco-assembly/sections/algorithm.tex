\section{An Adaptive ACO for the ALBP}
\subsection{Solution construction strategy}
In order to use ACO to solve the ALBP, the processing jobs can be seen as nodes on a graph which will be traversed by ants, also the connections between jobs and workstations can be seen as arcs on the graph.
The assignmnet of processing jobs to workstations can then be seen as an ant colony travels through the graph with the guidance of pheromone and heuristic information.

The solution construction is a key step in emplying ACO to solve ALBP and this paper constructs a feasible solution by gradually assigning processing jobs to corresponding workstations.
To this end, we define the following notations:
\begin{itemize}
	\item no-assigned task: a task that hasn't been assigned to any workstation yet
	\item available task: a task that hasn't yet been assigned but satisfy precedence constraints between tasks, also all of its preceeding tasks have been assigned
	\item assignable task: an available task that satisfies cycling constraints.
\end{itemize}

The feasible solution construction algorithm can then be described as follows:
\begin{enumerate}
	\item open a workstation
	\item identify the set of available tasks from all no-assigned tasks and the precedence constraints among tasks, if the resulting set is empty, go to step 7
	\item identify the set of assignable tasks from available tasks and the cycling constraint
	\item if the set of assinable tasks is empty, go to step 6
	\item select a task from the set of assinable tasks according to defined rules and assign it to the current workstation, go to step 2
	\item open a new workstation, go to step 3
	\item stop
\end{enumerate}

Using the way defined in the above algorithm to identify the set of assignable tasks, there always exists an optimal assinable task in the set.

One key characteristic of ACO is its utilization of pheromone feedback and heuristic information during its search for global optimality; therefore, the way of pheromone updating and heuristic information selection have a huge impact on the performance of ACO.
In this paper, we define $\tau_{ij}$ as the pheromone intensity on arc $(i,j)$ traversed by ants and represents the expectation of assigning task $i$ to workstation $j$.
The corrresponding heuristic informaiton is computed by static precedence rules.
In addition, the heuristic information of ACO consists of the maximal task completion time and maximal number of succedding tasks, and the visibility $\eta_i$ of task $i$ can be computed as
\begin{align}
	\eta_i = \frac{t_i}{C} + \frac{U_i}{max_{i = 1, 2, \cdots, N}U_i}
\end{align}
where $U_i$ is the number of succedding tasks of task $i$.


\subsection{Objective function}






\subsection{Pheromone update strategy}




