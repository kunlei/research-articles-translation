
\section{Introduction}
Assembly line is a widely used component in manufacturing enhancements, and various challenges emerge in its design and daily operations, among which the assembly line balancing problem (ALBP) exists as one of most significant ones.
It is noted that a small improvement towards ALBP can lead to significant efficiency enhancement and cost reduction \citep{salveson1};
On the other hand, ALBP is a classical NP-hard combinatorial optimization problem in which the complexity increases exponentially with the number of jobs and there yet exist polynomial-time algorithms to obtain optimal solutions in reasonable computation times \citep{kilincci2}.
Three solution strategies towards solving the ALBP can be found in the literature, namely, exact algorithms, heuristic algorithms and artificial intelligence algorithms.
Exact algorithms are able to find optimal solutions, but often require tremendous computation power and time.
Due to these limitations, they are mainly used to solve small-sized problems and can hardly be applied in real-world production systems \citep{scholl3, peeter4}.
Heuristic algorithms have received many attentions from the research community due to its implementation simplicity; however, they generally take long time to identify optimal solutions and also it is often hard to verify whether the solutions they found are optimal or not \citep{ponnambalam5}.
Artificial intelligence algorithms, including genetic algorithms, simulated annealing and tabu search, witness significant advancements in recent years and have been applied to solve the ALBP successfully \citep{azcan6, azcan7}.

Ant colony optimization (ACO) is an intelligent optimization algorithm proposed by Clolrni \citep{inproceedings8} in 1991 and has been applied to various combinatorial optimization problems.
\citet{bautista9} made the first attempt to solve a simple assembly line balancing problem using ant colony algorithms based on an ant system, but the optimization results were not optimal.
\citet{patrick10} obtained superior ACO performance on a number of benchmarking instances in solving a more complex assembly balancing problem that is characterized by mixed job types, stochastic processing times and parallel workstations.
\citet{bautista11} studied an assembly balancing problem with timing and spatial constraints and explored its solution method based on ant colony algorithms.

The aforementioned researches in solving assembly balancing problems using ACO suffer various performance issues when compared with other algorithms in the literature.
Some of them employ an oversimplified pheromone updating strategy that jeopardizes ACO's ability to converge to optimal solutions. 
Others use simple objective functions that often fail to correctly evaluate solution qualities, which results in inferior performance in identifying promising solutions.
To address these issues, we propose an adaptive ACO to solve the assembly line balancing problem.
It tries to avoid local optima by utilizing both external and historical information to adaptively adjust global pheromone evaporation factor in the process of ant path construction.
In addition, the algorithm incorporates balancing and smoothing factors in the objective function, which improves ACO's ability in identifying promising solutions.
Performance of the proposed algorithm is validated on benchmarking instances.
