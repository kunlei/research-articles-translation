\documentclass{article}
%\usepackage[utf8]{inputenc}
\usepackage[top=0.9in, bottom=1in, left=1in, right=1in]{geometry}
\usepackage{comment}
\usepackage{amsthm}
\usepackage{amssymb}% http://ctan.org/pkg/amssymb
\usepackage{pifont}% http://ctan.org/pkg/pifont
\usepackage{xcolor}
\usepackage{multirow}
\usepackage{hhline}
\newcommand{\cmark}{\ding{51}}%
\newcommand{\xmark}{\ding{55}}%
\usepackage{amsmath}
\usepackage{algorithm}
%\usepackage{algorithmic}
\usepackage{algpseudocode}
\usepackage{graphicx}
\usepackage{caption}
\usepackage{subcaption}
%\usepackage{lscape} 
\usepackage{pdflscape}
\usepackage{rotating}
\usepackage{multirow}
\usepackage{hhline}
\usepackage{comment}
\usepackage{float}
%cmark xmark
\usepackage{pifont}
\usepackage{color}
\usepackage{longtable}
\usepackage{multicol}
\usepackage{setspace}
\usepackage{authblk}
\providecommand{\keywords}[1]{\textbf{\textit{Key words: }} #1}
\doublespacing


\usepackage[round, sort]{natbib}

\title{An Adaptive Ant Colony Optimization for Solving Assembly Line Balancing Problem}
\author[1]{Fuping Deng}
\author[1]{Chaoyong Zhang}
\author[1]{Kunlei Lian}
\author[1]{Shaotan Xu}
\affil[1]{State Key Laboratory of Digital Manufacturing Equipment and Technology, Huazhong University of Science and Technology, Wuhan, 430074}
\date{}

\begin{document}

\maketitle

\begin{abstract}
A CCA was developed and modified to solve the problem of minimal candidates set.
The minimal candidate set was a minimal hitting set.
The modified CCA improved performance in initialization, assimilation and rebirth of original CCA by introducing a third type of country, independent country, to the population of countries maintained by CCA by introducing a third type of country, independent country, to the population of countries maintained by CCA.
Implementation details of the proposed CCA and modified colonial competitive algorithm (MCCA) were elaborated using an illustrative example.
The performance of the algorithms was analyzed, and the results by the MCCA were compared with DMDSE-Tree algorithm.
When 90\% of the minimal hitting sets are obtained, the MCCA has better efficiency.
Finally, the experimental results of certain system verify the effectiveness of the algorithm, which proves that this method can be applied in solving the minimal hitting set of combinatorial optimization problems for selection of equipment effectively.
\end{abstract}


\keywords{Business planning; colonial competitive algorithm (CCA); minimal hitting set; equipment selection}

\doublespacing


\section{Introduction}
Assembly line is a widely used component in manufacturing enhancements, and various challenges emerge in its design and daily operations, among which the assembly line balancing problem (ALBP) exists as one of most significant ones.
It is noted that a small improvement towards ALBP can lead to significant efficiency enhancement and cost reduction \citep{salveson1};
On the other hand, ALBP is a classical NP-hard combinatorial optimization problem in which the complexity increases exponentially with the number of jobs and there yet exist polynomial-time algorithms to obtain optimal solutions in reasonable computation times \citep{kilincci2}.
Three solution strategies towards solving the ALBP can be found in the literature, namely, exact algorithms, heuristic algorithms and artificial intelligence algorithms.
Exact algorithms are able to find optimal solutions, but often require tremendous computation power and time.
Due to these limitations, they are mainly used to solve small-sized problems and can hardly be applied in real-world production systems \citep{scholl3, peeter4}.
Heuristic algorithms have received many attentions from the research community due to its implementation simplicity; however, they generally take long time to identify optimal solutions and also it is often hard to verify whether the solutions they found are optimal or not \citep{ponnambalam5}.
Artificial intelligence algorithms, including genetic algorithms, simulated annealing and tabu search, witness significant advancements in recent years and have been applied to solve the ALBP successfully \citep{azcan6, azcan7}.

Ant colony optimization (ACO) is an intelligent optimization algorithm proposed by Clolrni \citep{inproceedings8} in 1991 and has been applied to various combinatorial optimization problems.
\citet{bautista9} made the first attempt to solve a simple assembly line balancing problem using ant colony algorithms based on an ant system, but the optimization results were not optimal.
\citet{patrick10} obtained superior ACO performance on a number of benchmarking instances in solving a more complex assembly balancing problem that is characterized by mixed job types, stochastic processing times and parallel workstations.
\citet{bautista11} studied an assembly balancing problem with timing and spatial constraints and explored its solution method based on ant colony algorithms.

The aforementioned researches in solving assembly balancing problems using ACO suffer various performance issues when compared with other algorithms in the literature.
Some of them employ an oversimplified pheromone updating strategy that jeopardizes ACO's ability to converge to optimal solutions. 
Others use simple objective functions that often fail to correctly evaluate solution qualities, which results in inferior performance in identifying promising solutions.
To address these issues, we propose an adaptive ACO to solve the assembly line balancing problem.
It tries to avoid local optima by utilizing both external and historical information to adaptively adjust global pheromone evaporation factor in the process of ant path construction.
In addition, the algorithm incorporates balancing and smoothing factors in the objective function, which improves ACO's ability in identifying promising solutions.
Performance of the proposed algorithm is validated on benchmarking instances.

\section{Dynamic Scheduling Problem Description}
Static scheduling problems involve $n$ jobs to be processed on $m$ machines and the scheduling plan can be determined after the processing order is decided for every job on all the machines.
However, in real-world manufacturing systems, this processing order needs to be rescheduled whenever new orders arrive, machines break down or raw materials get delayed.

Dynamic scheduling views the job manufacturing as a dynamic process in which jobs become available for processing continuously, followed by machine processing and exiting the manufacturing system after their processes are finished.
Dynamic events refer to entities that trigger the rescheduling of existing jobs due to changes in scheduling environment, and they can be classified into four categories:
\begin{itemize}
	\item job-related events, these include stochastic arrivals of jobs, undetermined job processing times, changes in delivery date, dynamic priorities and orders.
	\item machine-related evenst, these include machine break-down, limited capacity, machine deadlock and conflicts in manufacturing capacity.
	\item process-related events, these include process delay, quality negation and output unstability.
	\item other events, these include absent operators, delayed arrival or unexpected flaws of raw materials, and dynamic processing routes.
\end{itemize}

In the classical job shop static scheduling problems, the release times $r_i$ of all jobs are assumed to be zero and the objective is to minmize the makespan $C_{max}$, which is defined as the maximum completion time of all jobs, $C_{max} = \text{min}\{\text{max} C_i, i = 1, \cdots n\}$ where $C_i$ is the completion time of job $J_i$.
In real-world dynamic manufacturing environments, however, jobs become available for process sequentially, meaning that their release times $r_i$ are different and unpredictable. 
Since a job can only be processed after it becomes available, in dynamic scheduling problems, the maximum completion time of all jobs is determined by the completion time of the latest-released job.
Therefore, dynamic scheduling problems generally use the mean flow time of all jobs as the objective function instead of the maximum completion time of all jobs.

This paper considers two performance metrics often seen in dynamic manufacturing environments: 1) minimization of mean flow time $\bar{F}$ where $\bar{F} = \text{min}(\frac{1}{n} \times \sum_{i = 1}^n C_i - r_i)$ and $r_i$ and $C_i$ are the release time and completion time of job $J_i$, respectively.
2) minimization of the weighted objective of maximum completion time $C_{max}$ and total tardiness. This weighted objective gives higher priority to urgent jobs which are required to finish by their due dates, and minimizes the completion times of remaining jobs.
For a scheduling problem with $n$ jobs and $m$ machines, the objective can be defined as follows:
\begin{align}
	\text{min}(\text{max}C_i + \alpha \times (\sum\text{max}(0, C_j - D_J))), \ (i \in S_{J_1}, j \in S_{J_2})
\end{align}
where $\alpha$ is the weighted penalty coefficient, $S_{J_1}$ is the job set that is being processed and $S_{J_2}$ is the newly inserted urgent job set, $C_i$ is the completion time for job $i \in S_{J_1}$, and $D_j$ is the delivery date for urgent job $j \in S_{J_2}$, and $C_j$ is the completion time.


% \citet{brodeur} examine the daily linen delivery problem in the Jewish General Hospital in Montreal and model it as a PVRP. 
% A tabu search algorithm is used to solve the problem.
% \citet{wooseung} study a sales representatives routing problem for a lottery company in Missouri.
% Thirty-nine representatives are scheduled to visit 5043 ticket retailers periodically to check on product inventory, replenish supplies, collect returned tickets, clean point-of-sale counters and inspect equipments.
% \citet{maya2012} consider an assignment routing problem in which  disabled pupils are visited by assigned teaching assistants at their own schools. Pupils with various level of disability have different assistance frequency. 
% A solution approach based on an auction algorithm and a variable neighborhood search heuristic is proposed.

% Many solution approaches have been proposed since the pioneering work of \citet{belbod}.
% \citet{christofidesn1984} propose an exact formulation for PVRP but solve it via a heuristic.
% \citet{cordeaujf} propose a tabu search algorithm to solve the PVRP.
% Other recent algorithms proposed for PVRP include variable neighborhood search \citep{veravns}, ant colony optimization \citep{matos} and a genetic algorithm \citep{vidalga}.
% \citet{fortyyearspvrp} provide an extensive review on PVRP applications and solution methods.

% Service consistency in vehicle routing is first formulated in the Consistent Vehicle Routing Problem (ConVRP) \citep{convrp}. 
% The authors consider a routing problem inspired from the small package delivery industry where customers require one or more visits over a planning horizon of multiple days.
% Consistent services are enforced during the construction of vehicle routes using hard constraints and a record-to-record travel heuristic (ConRTR) is proposed to obtain near optimal solutions quickly.
% The concept of template routes is used in the heuristic to ensure a customer is visited by the same driver throughout the planning horizon and also to encourage better time consistency.
% Template routes consider only frequent customers that require service on more than one day.
% A route for day $d$ can be derived from the template route by removing all those customers who do not require service on day $d$ and inserting customers who require service on only day $d$.
% A tabu search heuristic (TTS) and adaptive large neighborhood search algorithm (TALNS) are developed in \citet{tarantilis} and \citet{TALN} to solve the ConVRP using the idea of template routes.

% In the Generalized Consistent Vehicle Routing Problem (GenConVRP) \citep{genconvrp}, strict driver consistency is relaxed by allowing multiple drivers to visit a customer and time consistency is incorporated into the objective function using a weighted sum approach.
% A large neighborhood search heuristic is proposed to solve this problem efficiently.
% \citet{Luo2015} investigate a multi-period VRP in which the maximum number of different drivers visiting a customer is limited and each customer is associated with a time window.
% A three-stage solution approach is proposed to solve this problem.
% The impact of imposing different levels of service consistency on operational cost is studied.

% \citet{Kris2016} introduce driver consistency to the classical Dial-a-ride Problem (DARP) by limiting the maximum number of different drivers assigned to service a user across the planning horizon.
% The authors suggest two mathematical formulations and a branch-and-cut solution method.
% A large neighborhood search heuristic is also developed to obtain near-optimal solutions quickly.
% Computational results show that enforcing only one driver visiting a customer may cause up to 27.98\% increase in travel cost, while the routing cost increase is no more than 5.80\% when at least two drivers are allowed per customer.


% \citet{contsp} study the Consistent Traveling Salesman Problem (ConTSP) over a multi-day planning horizon and consistent arrival times are imposed for frequent customers.
% The authors examine different mixed-integer linear programming formulations and develop a branch-and-cut framework with a new class of valid inequalities. 

% Research on the trade-offs between consistency objectives and other objectives appear in \citet{spicer}, \citet{mogenconvrp}, and \citet{imdls}.
% \citet{spicer} study a home healthcare nurse routing and scheduling problem with the objectives of minimizing total travel cost, maximizing nurse consistency and balancing nurse workload.
% A tabu search based multi-objective algorithm is developed to approximate the Pareto frontier.
% \citet{mogenconvrp} employ two $\epsilon$-constraint-based exact multi-objective solution approaches to analyze the trade-offs between the objective of travel cost minimization and the objectives of driver and time consistency maximization.
% A  multi-directional large neighborhood search heuristic is also proposed to solve large instances.
% It is shown that 70\% better time consistency can be achieved for the tested benchmark instances by at most 3.84\% increase in total travel cost.
% Most recently, \citet{imdls} investigate a multi-objective version of the ConVRP in which driver consistency and time consistency are optimized as individual objectives together with the minimization of travel cost.
% An improved multi-directional local search algorithm is proposed for general multi-objective optimization problems.
% It is shown that approximately 60\% better driver consistency and 75\% better time consistency can be achieved at the cost of 5\% increase in travel cost.



% The review of the literature indicates that there is currently a gap with respect to considering service consistency in the context of the periodic vehicle routing problem.
% We aim to fill this gap by studying the trade-offs between the traditional objective of minimizing total travel cost and the consistency objectives of maximizing driver consistency and time consistency.
% By analyzing the Pareto frontier obtained using various multi-objective algorithms, observations of the impact of improving service consistency on travel cost in the PVRP will be made to facilitate managerial decision making.

% \section{Problem description}\label{model}
% The multi-objective consistent periodic vehicle routing problem (MoConPVRP) studied in this paper can be defined on a complete directed graph $\mathcal{G} = (\mathcal{N}^0, \mathcal{A})$, where $\mathcal{N}^0 = \mathcal{N} \cup \{0\}$ with $\mathcal{N} = \{1, \dots, n\}$ representing the set of customers and $\{0\}$ indicating the depot,   and $\mathcal{A} = \{(i, j) | i, j \in \mathcal{N}^0, i \neq j\}$.
% Associated with each arc $(i, j) \in \mathcal{A}$ is a travel time $t_{ij}$ and triangle inequality is satisfied.
% A time horizon $\mathcal{D} = \{d_1, \dots, d_{|\mathcal{D}|}\}$ is considered and there are $|\mathcal{K}|$ homogeneous vehicles available at the depot.
% Each vehicle starts and ends its daily operations at the depot, and can only service a limited number of customers due to the restriction on its physical capacity $Q$ and maximum route duration $T$.
% In this paper, the terms \textit{driver} and \textit{vehicle} are used interchangeably.

% Each customer $i \in \mathcal{N}$ is associated with a service frequency $m_i$ and has a predetermined set of allowable service patterns $C_i$.
% A service pattern $r \in C_i$ specifies the days on which the customer $i$ is allowed to receive service.
% A  constant $w_{dr}$ is defined  such that $w_{dr} = 1$ if and only if day $d$ belongs to pattern $r$, and 0 otherwise.
% A non-negative demand $q_i$ and service duration $s_i$ are  known in advance.
% The problem is to choose a single allowable service pattern for each customer and determine a set of vehicle routes for each day of the planning horizon that are feasible with respect to capacity and route duration constraints.
% Each route must begin and end at the depot and each customer must be visited by exactly one vehicle on each day that he/she needs service.
% The objectives include minimizing total travel cost, minimizing the number of different drivers visiting a customer across the planning horizon and minimizing the maximum arrival time differential at a customer throughout the planning horizon.



% The problem can be modeled using the following decision variables:
% \begin{itemize}
% 	\item $x_{ijkd}$: binary variable indicating whether arc $(i, j)$ is traversed by vehicle $k$ on day $d$, $(i, j) \in \mathcal{A}$, $k \in \mathcal{K}$ and $d \in \mathcal{D}$
% 	\item $y_{ikd}$: binary variable indicating whether customer $i$ is visited by vehicle $k$ on day $d$, $i \in \mathcal{N}^0$, $k \in \mathcal{K}$ and $d \in \mathcal{D}$
% 	\item $z_{ir}$: binary variable indicating whether service pattern $r$ is chosen for customer $i$, $r \in C_i$ and $i \in \mathcal{N}$
% 	\item $a_{id}$: continuous variable describing the arrival time at customer $i$ on day $d$, $i \in \mathcal{N}^0$ and $d \in \mathcal{D}$
% \end{itemize}

% The following auxiliary variables are defined to facilitate the modeling and linearization of consistency objectives:
% \begin{itemize}
% 	\item $u_{ik}$: binary variable indicating whether customer $i$ is visited by vehicle $k$ over the planning horizon, $i \in \mathcal{N}^0$ and $k \in \mathcal{K}$
% 	\item $a_i^e$: continuous variable describing the earliest arrival time at customer $i$ over the planning horizon, $i \in \mathcal{N}^0$
% 	\item $a_i^l$: continuous variable describing the latest arrival time at customer $i$ over the planning horizon, $i \in \mathcal{N}^0$
% 	\item $u^{max}$: continuous variable representing the maximum number of different drivers visiting a customer
% 	\item $a^{max}$: continuous variable representing the maximum arrival time differential at a customer
% \end{itemize}


% Combining aspects of the PVRP formulation in \citet{cordeaujf} and the ConVRP formulation in \citet{convrp}, we formulate the MoConPVRP as follows:
% \allowdisplaybreaks
% \begin{align}
% 	\text{min} &\quad  \sum_{d \in \mathcal{D}} \sum_{k \in \mathcal{K}} \sum_{i \in \mathcal{N}^0} \sum_{j \in \mathcal{N}^0} t_{ij} x_{ijkd}, \label{ch3obj1} \\
% \text{min} &\quad u^{max}, \label{ch3obj2}\\
% \text{min} &\quad  a^{max}, \label{ch3obj3}\\
% 	\text{s.t.}
% 	&\quad   \sum_{r \in C_i} z_{ir} = 1, \; \forall i \in \mathcal{N}   \label{ch3cons1}  \\
% 	&\quad   \sum_{j \in\mathcal{N}^0} \sum_{k \in \mathcal{K}} x_{ijkd} - \sum_{r \in C_i} w_{dr} z_{ir} = 0,  \; \forall i \in \mathcal{N},  \; d \in \mathcal{D}    \label{ch3cons2}  \\ 
% 	&\quad  \sum_{k \in \mathcal{K}} y_{ikd} - \sum_{r \in C_i} w_{dr} z_{ir} = 0,  \; \forall i \in \mathcal{N},  \; d \in \mathcal{D}      \label{ch3cons3}  \\
% 	&\quad \sum_{i \in \mathcal{N}^0} x_{ijkd} = \sum_{i \in \mathcal{N}^0} x_{jikd} = y_{jkd}, \; \forall\ j \in \mathcal{N}^0, \; k \in \mathcal{K},  \; d \in \mathcal{D}        \label{ch3cons4}  \\
% 	&\quad  \sum_{j \in \mathcal{N}} x_{0jkd} \leq 1,  \; \forall k \in \mathcal{K},  \; d \in \mathcal{D}     \label{ch3cons5}  \\
% 	&\quad  y_{0kd} = 1,  \; \forall k \in \mathcal{K},  \; d \in \mathcal{D}       \label{ch3cons6}  \\
% 	&\quad  \sum_{i \in \mathcal{N}} q_{i} y_{ikd} \leq Q,  \; \forall k \in \mathcal{K},  \; d \in \mathcal{D}      \label{ch3cons7}  \\
% 	& \quad 0 \leq a_{id} + \sum_{r \in C_i} w_{dr} z_{ir}(s_{i} + t_{i0}) \leq T\sum_{r \in C_i} w_{dr} z_{ir}, \; \forall\ i  \in \mathcal{N}, \;  d \in \mathcal{D}    \label{ch3cons8}  \\
% 	&\quad  a_{0d} = 0,  \; \forall d \in \mathcal{D}       \label{ch3cons9}  \\
% 	& \quad a_{id} +  x_{ijkd}(s_{i} + t_{ij}) - T(1-x_{ijkd}) \leq a_{jd}, \; \forall\ d \in \mathcal{D}, \; k \in \mathcal{K}, \; i \in \mathcal{N}^0,  \; j \in \mathcal{N}    \label{ch3cons10}  \\
% 	& \quad a_{id} +  x_{ijkd}(s_{i} + t_{ij}) + T(1-x_{ijkd}) \geq a_{jd}, \; \forall\ d \in \mathcal{D}, \; k \in \mathcal{K}, \; i \in \mathcal{N}^0, \;  j \in \mathcal{N}   \label{ch3cons11}  \\
% 	& \quad u_{ik} \geq y_{ikd},  \; \forall i \in \mathcal{N},  \; k \in \mathcal{K},  \; d \in \mathcal{D} \label{ch3cons12}  \\
% 	& \quad \sum_{k \in \mathcal{K}} u_{ik} \leq u^{max},  \; \forall i \in \mathcal{N}   \label{ch3cons13}  \\
% 	& \quad a_i^l \geq a_{id} \geq a_i^e,  \; \forall i \in \mathcal{N},  \; d \in \mathcal{D}   \label{ch3cons14}  \\
% 	& \quad a_i^l - a_i^e \leq a^{max},  \; \forall i \in \mathcal{N}    \label{ch3cons15}  \\
% 	& \quad  x_{ijk}^d \in \{0,1\}, \; \forall\ i \in \mathcal{N}^0, \;  j \in \mathcal{N}^0, \;  k \in \mathcal{K}, \; d \in \mathcal{D}     \label{ch3cons16}  \\
% 	& \quad y_{ikd} \in \{0,1\}, \; \forall\ i \in \mathcal{N}, \;  k \in \mathcal{K}, \; d \in \mathcal{D}   \label{ch3cons17}  \\
% 	& \quad z_{ir} \in \{0, 1\}, \; \forall i \in \mathcal{N},  \; r \in C_i    \label{ch3cons18} \\
% 	& \quad a_{id} \geq 0, \; \forall\ i \in \mathcal{N}^0, \;  d \in \mathcal{D}    \label{ch3cons19}  \\
% 	& \quad u_{ik} \geq 0, u^{max} \geq 0, \; \forall\ i \in \mathcal{N}, \;  k \in \mathcal{K}    \label{ch3cons20}  \\
% 	& \quad a_i^e, a_i^l \geq 0, a^{max} \geq 0 \; \forall\ i \in \mathcal{N}\    \label{ch3cons21} 
% \end{align}

% Objective (\ref{ch3obj1})  minimizes the total travel distance of all vehicles on all days of the planning horizon.
% Objective (\ref{ch3obj2}) minimizes the maximum number of different drivers that visit any one customer.
% Objective (\ref{ch3obj3}) minimizes the maximum arrival time differential experienced by any one customer.
% Constraint set (\ref{ch3cons1}) ensures that each customer is assigned one and only one allowable service pattern.
% Constraint set (\ref{ch3cons2}) ensures that the days on which a customer receives visits are those days appearing in the service pattern selected for the customer.
% Constraint set (\ref{ch3cons3}) ensures that each required visit to a customer is performed by exactly one vehicle.
% Constraint set (\ref{ch3cons4}) makes sure that each customer has only one predecessor and successor whenever it requires service.
% Constraint sets (\ref{ch3cons5}) and (\ref{ch3cons6}) ensure that each vehicle departs the depot at most once per day.
% The vehicle capacity and route duration constraints are given in constraint sets (\ref{ch3cons7}) and (\ref{ch3cons8}), respectively.
% Constraint sets (\ref{ch3cons9}), (\ref{ch3cons10}) and (\ref{ch3cons11}) aim to compute vehicle arrival times at customers and also serve to eliminate subtours.
% Constraint sets (\ref{ch3cons12}) and (\ref{ch3cons13}) compute the maximal number of different drivers visiting a customer.
% Constraint sets (\ref{ch3cons14}) and (\ref{ch3cons15}) compute the maximum arrival time differential over all customers.
% The variable types are given in remaining constraint sets.
% Note that objectives \eqref{ch3obj2}--\eqref{ch3obj3},  constraint sets \eqref{ch3cons3} and \eqref{ch3cons12}--\eqref{ch3cons15}, and variables \eqref{ch3cons16}--\eqref{ch3cons21} are newly introduced in this paper. The objective \eqref{ch3obj1} and  constraint sets \eqref{ch3cons1},  \eqref{ch3cons2} and \eqref{ch3cons4}--\eqref{ch3cons11} are taken from \citet{cordeaujf} and \citet{convrp}.



% \section{Solution approach}\label{solution}
% This section briefly describes the multi-objective algorithms employed in this paper to approximate the Pareto frontier of the  MoConPVRP.
% The solution generation method and local search operators for the three objectives will be detailed in following sections.
% The term \textit{solution} used from this point forward refers to a routing plan consisting of a set of vehicle routes for each day of the planning horizon such that each customer is assigned an allowable service pattern, all customer demands are satisfied and vehicle capacity and maximum route duration constraints are respected.

% \subsection{Multi-objective optimization algorithms}
% Using exact solution approaches to obtain the Pareto frontier of MoConPVRP is time-consuming since these methods rely on iteratively solving single-objective optimization problems.
% Given that PVRP reduces to classical VRP when the planning horizon is set to one, PVRP is also NP-hard and therefore the time required to obtain the Pareto frontier of MoConPVRP is prohibitive.
% In this paper, we use seven multi-objective algorithms to approximate the Pareto frontier of MoConPVRP.
% The rationale  is that different algorithms have different strengths in approximating the true Pareto frontier and a better approximation of the true Pareto frontier can be obtained by utilizing a set of multi-objective algorithms together.
% Also, it provides us an opportunity to evaluate the performance of various multi-objective algorithms for the MoConPVRP.
% The seven multi-objective algorithms include 
% the Multi-directional Local Search (MDLS), 
% the Improved Multi-directional Local Search (IMDLS), 
% the Multi-objective Evolutionary Algorithm based on Decomposition (MOEA/D), 
% the Nondominated Neighbor Immune Algorithm (NNIA),
%  the Strength Pareto Evolutionary Algorithm 2 (SPEA2), 
%  the Nondominated Sorting Genetic Algorithm II (NSGAII) and the Nondominated Sorting Genetic Algorithm III (NSGAIII) \citep{mdls, imdls, moead, nnia, spea2, nsga2, nsga3}.
% \color{red}The effectiveness of this set of algorithms for a wide range of multi-objective optimization problems has been demonstrated (see, e.g., \citet{Ardakan2018, Ding2018, Zhu2018, Biswas2018, Zhang2016, Molenbruch2017, Mkaouer2015},  …etc.).
% \color{black}




% MDLS is an algorithmic solution framework for general multi-objective optimization problems \citep{mdls}.
% It is inspired by the concept of Pareto dominance.
% Starting from any solution $x$ in the solution space, a neighboring solution $x'$ is desirable if it is not dominated by $x$, which means that $x'$ is either dominating $x$ or non-comparable with $x$.
% The concept of Pareto dominance simply requires $x'$ be better than $x$ on at least one objective in order for $x'$ to dominate $x$.
% Therefore, it is enough to apply local search operators to each objective individually to find such a neighbor solution $x'$.
% MDLS starts with an initial set of non-dominated solutions $\mathcal{F}$ and goes through three steps in each subsequent iteration: (1) selecting a solution, (2) applying local search on this solution for each objective to obtain a new solution corresponding to each objective and (3) updating the set of non-dominated solutions $\mathcal{F}$ using the newly generated solutions.
% \citet{imdls} introduce three new features to the original MDLS and propose the IMDLS.
% Specifically, in IMDLS,  the size of $\mathcal{F}$ is bounded by a fixed number and all the non-dominated solutions in $\mathcal{F}$ are selected for local search in each iteration.
% If the size of $\mathcal{F}$ exceeds the input bound, a truncation procedure is applied on $\mathcal{F}$ using a crowding distance based selection rule.

% MOEA/D,  proposed by \citet{moead}, utilizes a decomposition strategy to transform a multi-objective optimization problem into a number of scalar optimization problems and optimize them simultaneously.
% In MOEA/D, an initial population of solutions are first generated and each of them is assigned to a predefined weight vector.
% All the neighbors of a solution in the population are then identified based on their corresponding weight vectors. 
% In each iteration, each solution $x$ in the current population is examined  and a new solution $y$ is obtained using two of $x$'s neighboring solutions.
% The new solution $y$ is then improved using local search operators to get $y'$ which is used to update all the neighboring solutions of $x$ based on their corresponding weight vectors.
% The authors propose three decomposition approaches, including the weighted sum approach, the Tchebycheff approach and the penalty-based boundary intersection approach.
% Each of the decomposition strategies is used to convert the various objectives to a scalar value and the corresponding MOEA/D variants are named MOEA/D-WS, MOEA/D-TCH and MOEA/D-PBI, respectively.

% NNIA is a multi-objective optimization algorithm inspired from the immune system's ability to adapt its B-cells to new types of antigens \citep{nnia}.
% The term \textit{antibody} is used to represent a solution.
% The algorithm maintains a set of non-dominated antibodies, namely, the dominant population $D$.
% In each iteration $t$, a fixed number of antibodies are selected from the dominant population $D_t$ to create an active population $A_t$.
% Each active antibody is then cloned a number of times based on its crowding distance value.
% The resulting clone population $C_t$ is subject to recombination and hypermutation, and $C_t'$ denotes the new clone population.
% The $C_t'$ is then combined with $D_t$ to identify  the new dominant population $D_{t + 1}$.


% SPEA2 is proposed by \citet{spea2} to improve the original SPEA.
% It maintains a solution population $P$ and an archive population $\bar{P}$ throughout its search process.
% SPEA2 starts with generating an initial population $P_0$ and empty archive $\bar{P_0}$,  and in each following iteration $t$, all non-dominated solutions are first identified from $P_t$ and $\bar{P_t}$ and saved in $\bar{P}_{t+1}$. 
% Binary tournament selection is then used to fill the mating pool based on $\bar{P}_{t + 1}$. 
% The new population in iteration $t + 1$, $P_{t + 1}$, is obtained by applying recombination and mutation operators to the mating pool. 
% This process repeats until a maximum number of iterations are reached.


% NSGAII is proposed by \citet{nsga2} and starts with creating a random parent population $P_0$ of size $N$. 
% An offspring population $Q_0$ of the same size is generated by applying  binary tournament selection, recombination and mutation operators on $P_0$.
% In each subsequent iteration $t$, $P_t$ and $Q_t$ are first combined to form $R_t$ which is sorted into different dominance levels ($F_1, F_2, \dots$).
% The new population $P_{t + 1}$ is then generated by selecting these levels one at a time until the size of $P_{t + 1}$ reaches $N$.
% If $|P_{t + 1}|$ exceeds $N$ by including a dominance level $F_l$, only those solutions in $F_l$ with best crowding distance values are accepted.
% The new offspring population $Q_{t + 1}$ is generated using crowded-comparison based binary tournament selection, recombination and mutation on solutions from $P_{t + 1}$.
% NSGAIII \citep{nsga3} follows the same framework with NSGAII.
% The difference is how solutions from dominance level $F_l$ are selected when creating $P_{t + 1}$.
% Instead of using a crowding distance based selection operator as in NSGAII, NSGAIII employs a more complex rule to identify the solutions in $F_l$ to enter $P_{t + 1}$.
% Let $S_t = \sum_{i = 1}^{l}F_i$, where the objective values of solutions in $S_t$ are first normalized.
% A reference set $Z^r$ is created and each solution $s \in S$ is associated with a reference point in $Z^r$.
% A niche count $\rho_j$ is computed for each reference point $j \in Z^r$.
% Finally, solutions to enter $P_{t + 1}$ are determined based on $\rho$.



% \subsection{Starting solution generation}\label{ch3solgen}
% All of the multi-objective algorithms used in this paper require a starting set of randomly generated solutions for the MoConPVRP.
% The solution generation process starts by assigning an allowable service pattern to each customer sequentially and the order in which customers are considered for assignment is randomized.
% For a customer being considered for service pattern assignment, the service pattern that will balance the total demand on each day is chosen.
% To this end, for a service pattern,  the average daily total demand across the planning horizon is first noted if this service pattern is chosen for this customer.
% Then the total daily demand difference from this average demand is computed.
% The service pattern that results in the smallest demand difference will be chosen for this customer.

% After the service pattern for each customer is determined, the resulting vehicle routing problem is solved on each day using a cluster-first route-second approach.
% On each day of the planning horizon, the Sweep heuristic \citep{sweep} for VRP is used to order the customers with respect to the angles they make with the horizontal line and the depot.
% Then, the Next Fit bin packing heuristic \citep{nextfit} is used with this ordering to assign customers to vehicles. 
% A new vehicle will be used if the vehicle capacity constraint is violated.
% All of the remaining customers will be assigned to the last vehicle if no more vehicles are available.
% In this way, all but the last vehicle will satisfy the vehicle capacity constraint.
% The actual route operated by a vehicle is determined using Farthest Insertion  \citep{farthestinsertion}.
% To initialize the heuristic, the convex hull for all customers assigned to this vehicle and the depot serves as the initial route.
% Note that the resulting vehicle route may violate the maximum route duration constraint. 

% With any feasible solution, its total travel cost can be computed by summing up the distance traveled by all vehicles across the planning horizon.
% The driver consistency objective value can be determined by identifying the maximum number of different drivers visiting a customer across the planning horizon.
% The time consistency value can be computed by determining the maximum arrival time differential at a customer across the planning horizon.
% Objective values of infeasible solutions will be penalized using formula \eqref{ch3penobj} in Section \ref{ch3lsobj1}.

% \color{red}
% The above procedure generates a candidate solution for the starting population of the chosen multi-objective algorithms.
% This solution is accepted if it is feasible. 
% Otherwise, the local search operator for the travel cost objective described in the following section is applied in hope to reach feasibility, if successful, the feasible solutions will be added to the population.
% This process repeats until the desired population size is reached.


% \color{black}
 

% \subsection{Local search for the  travel cost objective}\label{ch3lsobj1}
% This section describes a tabu search heuristic used in this paper to minimize the total travel cost.
% Algorithm \ref{ch3tabusearch1} shows the framework of the heuristic.
% Infeasible solutions are allowed during the search process and the vehicle capacity and route duration constraint violations are penalized using coefficients $\gamma$ and $\delta$, respectively.
% Specifically, the penalized travel distance of a vehicle route $k$ on day $d$ is computed as
% \begin{align}
% 	\bar{f}_{TD}(k, d) = f_{TD}(k, d) + \gamma * \max(Q_{k,d} - Q, 0) + \delta * \max (T_{k, d} - T, 0), \label{ch3penobj}
% \end{align}
% %\begin{align}
% %	\bar{f}_{TD}(k, d)  & \quad = f_{TD}(k, d) \label{ch3penobj} \\
% %	 & \quad + \gamma * \max(Q_{k,d} - Q, 0) \nonumber \\
% %	  & \quad + \delta * \max (T_{k, d} - T, 0) \label{ch3:tdeval} \nonumber
% %\end{align}
% where $f_{TD}(k, d)$ is the travel distance of the route $k$ on day $d$, and $Q_{k, d}$ and $T_{k, d}$ are the total load and travel time of vehicle $k$ on day $d$, respectively.
% The penalty coefficient $\gamma$ is applied to any vehicle whose total load exceeds its capacity ($Q_{k, d} > Q$). 
% Similarly, the penalty coefficient $\delta$ is applied to any vehicle whose total travel time exceeds its limit ($T_{k, d} > T$). 

% \begin{algorithm}[H]
% 	\caption{Tabu search heuristic for minimizing total travel cost}
% 	\label{ch3tabusearch1}
% 	\begin{algorithmic}[1]
% 		\State \textbf{Input: } initial solution $\pi_0$
% 		\State initialize coefficients $\gamma$,  $\delta$ and $\lambda$
% 		\State initialize tabu list and frequency list and let iteration $\rho = 0$
% 		\State let current solution $\pi_\rho = \pi_0$ and let $\pi^{*}$ denote the best solution, $\pi^* = \pi_\rho$
% 		\State apply local search operator on vehicle routes on each day separately
% 		\Repeat
% 		\State construct a list of candidate neighboring solutions of $\pi_\rho$
% 		\State choose the best admissible solution $\pi'$ from the list
% 		\State let new current solution $\pi_{\rho} = \pi'$
% 		\State update $\pi^*$ if $\bar{f}_{TD}(\pi_{\rho}) < \bar{f}_{TD}(\pi^*)$
% 		\State update tabu list and frequency list
% 		\State update penalty coefficients
% 		\State apply local search operator on daily vehicle routes every $\xi$ iterations
% 		\State let $\rho = \rho + 1$
% 		\Until{stopping criteria is met}
% 	\end{algorithmic}
% \end{algorithm}


% The proposed heuristic starts with a randomly generated solution $\pi_0$ and it is evaluated using equation (\ref{ch3penobj}) to get its total travel cost $\bar{f}_{TD}(\pi_0)$.
% In each subsequent iteration $\rho$, a list of candidate solutions can be obtained from the current solution $\pi_\rho$ by selecting a single customer in $\pi_\rho$ and changing its current service pattern to another randomly selected allowable service pattern.
% Thus the total number of candidate solutions generated in each iteration equals the total number of customers having more than one allowable service pattern.
% To change the service pattern of a customer, its old and new service patterns are compared on each day $d$ sequentially: if the customer requires service on day $d$ only in the old service pattern, the visit to this customer is removed from the solution; if the customer requires service on day $d$ only in the new service pattern, the visit to this customer is inserted to the solution with least penalized travel distance increase; no change is made for the days on which the customer requires visits in both the old and new service patterns.


% A tabu list is initialized such that each allowable service pattern of a customer is assigned a tabu status which is an integer value initialized as 0 at the beginning of the algorithm.
% In iteration $\rho$, to decide whether a candidate solution $\pi'$ should be accepted as the new current solution $\pi_{\rho + 1}$, the only customer $c$ with changed service pattern in $\pi'$ compared to $\pi_\rho$  is identified and the tabu status of the service pattern for $c$  in $\pi'$ is checked.
% If the tabu status value is smaller than the current iteration number, the candidate solution $\pi'$ is accepted as the new current solution $\pi_{\rho + 1}$; otherwise, the $\pi'$ is still accepted as the  $\pi_{\rho + 1}$ if the aspiration criterion is satisfied, that is,  the $\pi'$ is better than the best solution $\pi^*$ encountered since the beginning of the algorithm. 
% If $\pi'$ is accepted as  $\pi_{\rho + 1}$, the tabu status of the service pattern of $c$ used in $\pi_\rho$ is updated as the summation of the current iteration number and the tabu length $\zeta$.
% In this way, the old service pattern is forbidden to be selected for customer $c$ for the next $\zeta$ number of iterations unless aspiration criteria is met.


% A frequency list is also created to record the number of times an allowable service pattern is chosen for a customer throughout the search process.
% This is to decrease the likelihood that a candidate solution with worse objective value than the incumbent will be accepted if the service pattern in the candidate solution has been frequently used.
% Every time a candidate solution $\pi'$ is accepted as the new current solution $\pi_{\rho + 1}$, the frequency $\mu \in \mathbb{N}$ of the new service pattern $r$ of the aforementioned customer $c$ in $\pi'$ is increased by 1.
% This service pattern selection frequency is used to penalize a candidate solution $\pi'$ whose penalized objective value is bigger than that of the current solution $\pi_\rho$ \citep{cordeaujf}.
% In other words, if $\bar{f}_{TD}(\pi') > \bar{f}_{TD}(\pi_\rho)$, the frequency-based penalized objective value is $g(\pi') = \bar{f}_{TD}(\pi') + \lambda \bar{f}_{TD}(\pi') \sqrt{n}\mu$, where $\lambda$ is the penalty coefficient and $n$ the total number of customers; otherwise, $g(\pi') = \bar{f}_{TD}(\pi')$ if $\bar{f}_{TD}(\pi') \leq \bar{f}_{TD}(\pi_\rho)$.
% In each iteration, all the candidate solutions are sorted in non-decreasing order of $g(\pi')$ and they are then checked sequentially to decide whether to become the new current solution $\pi_{\rho + 1}$.

% The penalty coefficients $\gamma$ and $\delta$ are updated at each iteration based on the feasibility of the current solution.
% Specifically, after a new current solution $\pi_{\rho}$ is obtained, its feasibility is checked regarding vehicle capacity and maximum route duration constraints. 
% If $\pi_{\rho}$ is feasible, the penalty coefficients $\gamma$ and $\delta$ are updated by multiplying by factor $\sigma$; otherwise, they are updated by adding a factor $\tau$.
% The parameter $\sigma$ is a relatively small number in order to decrease the penalty applied on infeasible routes, while the $\tau$ represents a small increment to the penalty coefficients.

% In Algorithm \ref{ch3tabusearch1}, a local search operator is applied on daily vehicle routes periodically.
% %For instances with only one vehicle available, the 2-opt heuristic is used \citep{twoopt}. 
% It is a  tabu search heuristic that aims to optimize the vehicle routes on any given day.
% Algorithm \ref{ch3tabusearch2} shows the framework of the tabu search heuristic.
% In this heuristic, a solution refers to the set of vehicle routes on day $d$.
% In each iteration $\rho$, a candidate solution $\pi'$ is created from the current solution $\pi_{\rho}$ by removing a customer $c$ from its current vehicle and inserting it into another vehicle.
% The total number of candidate solutions generated in each iteration equals the product of the total number of customers requiring service on day $d$ and the total number of vehicles minus 1.

% A tabu status value is assigned to each customer-vehicle combination such that a customer $c$ is forbidden to be inserted into a vehicle $k$ for  $\zeta_d$ iterations  if $c$ is removed from vehicle $k$.
% All tabu status values are initialized to 0 at the beginning of the algorithm.
% The same candidate acceptance rule and aspiration criterion as in Algorithm \ref{ch3tabusearch1} are used here.
% Similarly, a frequency list is created to record the number of times a customer $c$ is inserted into a vehicle $k$ throughout the search process.
% The same frequency update rule as in Algorithm \ref{ch3tabusearch1} is used here.
% Also, candidate solutions with worse penalized objective values are further penalized using the frequency-based approach explained previously in this section.
% Note that in Algorithm \ref{ch3tabusearch2}, the penalty coefficients $\gamma_d$ and $\delta_d$ are set as constant values.

% \begin{algorithm}
% 	\caption{Tabu search heuristic for minimizing total travel cost on day $d$}
% 	\label{ch3tabusearch2}
% 	\begin{algorithmic}[1]
% 		\State \textbf{Input: } initial solution $\pi_0^d$
% 		\State initialize coefficients $\gamma_d$,  $\delta_d$ and $\lambda_d$
% 		\State initialize tabu list and frequency list and let iteration $\rho = 0$
% 		\State let current solution $\pi_\rho^d = \pi_0^d$ and let $\pi_d^{*}$ denote the best solution, $\pi_d^* = \pi_\rho^d$
% 		\Repeat
% 		\State construct a list of candidate neighboring solutions of $\pi_\rho^d$
% 		\State choose the best admissible solution $\pi'_d$ from the list
% 		\State let new current solution $\pi_{\rho}^d = \pi'_d$
% 		\State update $\pi_d^*$ if $\bar{f}_{TD}(\pi_{\rho}^d) < \bar{f}_{TD}(\pi^*_d)$
% 		\State update tabu list and frequency list
% 		\State let $\rho = \rho + 1$
% 		\Until{stopping criteria is met}
% 	\end{algorithmic}
% \end{algorithm}

% %
% %\newpage
% %\begin{multicols}{3}
% %	\begin{itemize}
% %		\item $\alpha$: 
% %		\item $\beta$: 
% %		\item $\gamma$:  capacity violation coefficient
% %		\item $\delta$: duration violation coefficient
% %		\item $\epsilon$: 
% %		\item $\zeta$: tabu length
% %		\item $\eta$: 
% %		\item $\theta$
% %		\item $\iota$
% %		\item $\kappa$
% %		\item $\lambda$: frequency penalty term
% %		\item $\mu$: frequency
% %		\item $\nu$
% %		\item $\xi$: local search interval
% %		\item $\pi$: solution
% %		\item $\rho$: iteration
% %		\item $\sigma$: penalty decrease factor
% %		\item $\tau$:  penalty increase factor
% %		\item $\upsilon$: 
% %		\item $\phi$
% %		\item $\chi$
% %		\item $\psi$
% %		\item $\omega$
% %		\item $\Gamma$: 
% %		\item $\Delta$
% %		\item $\Lambda$: 
% %		\item $\Phi$: 
% %		\item $\Psi$: 
% %		\item $\Sigma$
% %		\item $\Theta$
% %		\item $\Upsilon$
% %		\item $\Xi$
% %		\item $\Omega$: 
% %	\end{itemize}
% %	\columnbreak
% %	\begin{itemize}
% %		\item $a$: decision variable
% %		\item $b$: 
% %		\item $c$: 
% %		\item $d$: day
% %		\item $e$: 
% %		\item $f$: objective function
% %		\item $g$: objective function
% %		\item $h$: 
% %		\item $i$: 
% %		\item $j$: 
% %		\item $k$: vehicle
% %		\item $l$: 
% %		\item $m$: 
% %		\item $n$: number of customer
% %		\item $o$: 
% %		\item $p$: 
% %		\item $q$: demand
% %		\item $r$: pattern
% %		\item $s$: solution / service time
% %		\item $t$: travel time
% %		\item $u$: decision variable
% %		\item $v$: vehicle
% %		\item $w$: indicator
% %		\item $x$: decision variable
% %		\item $y$: decision variable
% %		\item $z$:  decision variable
% %	\end{itemize}
% %	\columnbreak
% %	\begin{itemize}
% %		\item $A$: 
% %		\item $B$:  
% %		\item $C$: 
% %		\item $D$: 
% %		\item $E$: 
% %		\item $F$: 
% %		\item $G$: 
% %		\item $H$: 
% %		\item $I$: 
% %		\item $J$: 
% %		\item $K$: 
% %		\item $L$: 
% %		\item $M$: 
% %		\item $N$: 
% %		\item $O$: 
% %		\item $P$: 
% %		\item $Q$: capacity
% %		\item $R$: 
% %		\item $S$: 
% %		\item $T$: maximum travel time
% %		\item $U$: 
% %		\item $V$: 
% %		\item $W$: 
% %		\item $X$: 
% %		\item $Y$: 
% %		\item $Z$: 
% %	\end{itemize}
% %\end{multicols}
% %\newpage



% \subsection{Local search for the consistency objectives}
% %For the time consistency objective, the same heuristic proposed in \citet{imdls} to achieve time consistency in the context of the MoConVRP is used.
% %The heuristic is not adapted to explicitly address the service pattern decisions present in the MoConPVRP.
% %This is due to the difficulty of identifying an alternative allowable service pattern that can improve the time consistency objective.
% %The solution techniques tailored to addressing this challenge are saved for future work.

% The local search operators used to optimize service consistency explicitly treat service pattern selection decisions.
% These operators are described in this section.
% The heuristic for improving driver consistency will first randomly choose a customer with the worst driver consistency objective value and all the required visits to this customer will be removed from the solution.
% If there is only one service pattern available for this customer, all the vehicle routes on all the days on which it requires service will be checked to determine whether this customer can be feasibly inserted without violating vehicle capacity and route duration constraints.
% Then all available vehicles will be checked sequentially to compute the maximum number of days they can service this customer and the total travel cost increase that would result.
% The vehicle $v^*$ that can visit this customer most frequently with least travel cost increase will be selected to visit this customer whenever it is feasible to do so.
% For the days on which the customer visit requests are not satisfied, all of the vehicles are checked again using a similar procedure and the best vehicle is selected.
% This process repeats until all of the required visits to this customer are satisfied.

% If there are multiple service patterns available for this customer, all of the vehicle routes on all days across the planning horizon will be checked to determine whether this customer can be feasibly inserted without violating the vehicle capacity and route duration constraints.
% For each available service pattern, the algorithm determines the minimum number of different drivers that can satisfy all of the service requirements of this customer and the total travel cost increase associated with this service pattern.
% This determination process follows the same steps as in the previous one service pattern scenario.
% Then the service pattern with the best driver consistency objective value for this customer will be selected.
% The above removal and reinsertion process will be repeated until no improvement in the driver consistency objective can be made.
% \color{red}
% The above procedure is repeated until no improvement is observed after 50 iterations.
% \color{black}


% %\subsection{Local search for the time consistency objective}

% The local search operator used to optimize time consistency also considers service pattern selection decisions.
% It starts with identifying the customer $c^*$ with the worst time consistency objective value.
% \color{red}
% In case of multiple drivers having the same time consistency value, a customer is chosen randomly.
% \color{black}
% Then, the visit to customer $c^*$ which will be removed is identified as follows.  First, the average of the earliest and latest arrival time to customer $c^*$ is denoted $\bar{a}$ and the median arrival time to customer $c^*$ is denoted $med(a)$.  Then, if $med(a)>\bar{a}$, the visit with the earliest arrival time will be removed, and otherwise, the visit with the latest arrival time will be removed.  Denote the day of this visit $d^*$. 
% The process for reinserting the visit on day $d^*$ for customer $c^*$ begins by identifying all the alternative service patterns, saved in list $\mathcal{L}_p$, for customer $c^*$ that allow reinserting this visit to other days than $d^*$ without violating feasibility. 


% Consider two scenarios: if the list $\mathcal{L}_p$ is empty, it means that the removed visit can be reinserted back to the solution only on day $d^*$.
% In this case, all the route on day $d^*$ are sorted into a list $\mathcal{L}$ in nondecreasing order of the angle between the location of customer $c^*$ and the centroid of the route, taking the depot at the origin.
% Then the routes in $\mathcal{L}$ are examined sequentially to verify whether it is possible to improve the time consistency objective by reinserting the removed visit into a feasible insertion location on the route.
% If there exist multiple such locations within the route, the operator selects the one that improves time consistency the most.
% The removed visit is inserted into the first route in list $\mathcal{L}$ that contains such location.
% If the list $\mathcal{L}_p$ is not empty, the alternative service patterns in $\mathcal{L}_p$ are checked sequentially using the procedure defined in the first scenario. 
% Note that the removed visit can be reinserted on a day other than $d^*$ if a different service pattern is considered.
% The service pattern that results in the best time consistency is selected.
% \color{red}
% Similar to local search operator for the driver consistency objective, the above procedure is repeated until no improvement is observed after 50 iterations.
% \color{black}


% \subsection{Integration of local search operators with multi-objective algorithms}
% \color{red}
% This section describes the ways in which the three local search operators integrate with the seven multi-objective algorithms.
% In this paper, MDLS and IMDLS use the same strategy to incorporate the local search operators since they share the same workflow.
% They both have the key step of selecting candidate solution(s) from the current non-dominated solutions $\mathcal{F}$ to apply local search operators to improve its objectives. 
% For a candidate solution, each of the local search operators is applied separately to create a new solution.
% The resulting solution, if feasible, will be saved as a new solution to be considered to enter the next population.

% In MOEA/D, a new solution $y$ is created for each solution $x$ using its two neighboring solutions in the current population.
% The three local search operators are then apply sequentially on the new solution $y$ to get $y'$.
% The local search operator for the travel cost objective is applied first, followed by the operator for driver consistency and time consistency.
 
% In NNIA, the three local search operators are integrated into the recombination and hypermutation steps to create new clone population $C_t'$ based on starting clone population $C_t$.
% In the step of recombination, the local search operators are applied sequentially following the same order as in MOEA/D on a new solution created from two parent solutions.
% In the step of hypermuation, only the local search operators for driver consistency and time consistency are applied sequentially to serve as the hypermutation operator.

% In SPEA2, NSGAII and NSGAIII, the three local search operators are used sequentially following the same order as in MOEA/D on a new solution created from the recombination operator.
% Together they server as the mutation operator in these algorithms.



% \color{black}



% \section{Computational experiments and analysis}\label{results}
% In this section, the benchmark instances used in this paper to evaluate the performance of various multi-objective algorithms for solving MoConPVRP are first introduced.
% Next, the three metrics used to compare the performance of the various algorithms are described.
% Finally, the results of the computational study are presented and a trade-off analysis between the objectives of cost minimization and consistency maximization is provided.

% \subsection{Benchmark instances and experiment design}
% A total of 26 instances are taken from the literature \citep{cordeaujf} to validate the performance of the various multi-objective algorithms and to study the trade-offs between the objectives of cost minimization and consistency maximization.
% There exist some instances in the literature in which each customer requires service only once and they are excluded from the computational study in this paper.
% Table \ref{ch3instsum} summarizes the characteristics of these instances.
% The total number of customers $|\mathcal{N}|$, the total number of days $|\mathcal{D}|$ in the planning horizon, the number of vehicles $|\mathcal{K}|$ available and the best known travel distance objective values BKS from the literature are given for each instance.




% %\allowdisplaybreaks
% %\begin{table}
% %	\centering
% %	\caption{Summary of PVRP instances}
% %	\label{ch3instsum}
% %	\begin{tabular}{lllll|lllll}
% %		\hline
% %		Name & $|\mathcal{N}|$ & $|\mathcal{D}|$ & $|\mathcal{K}|$ & Service Frequency  & Name & $|\mathcal{N}|$ & $|\mathcal{D}|$ & $|\mathcal{K}|$ & Service Frequency \\
% %		\hline
% %		p01  & 51           & 2       & 3           & All One day   &    p17  & 40           & 4       & 4           & Mix \\
% %		p02  & 50           & 5       & 3           & Mix                &  	p18  & 76           & 4       & 4           & Mix   \\
% %		p03  & 50           & 5       & 1           & All One day   &  p19  & 112          & 4       & 4           & Mix             \\
% %		p04  & 75           & 2       & 5           & All One day   &   p20  & 184          & 4       & 4           & Mix          \\
% %		p05  & 75           & 5       & 6           & Mix               &   p21  & 60           & 4       & 4           & Mix    \\
% %		p06  & 75           & 10      & 1           & All One day &  	p22  & 114          & 4       & 6           & Mix            \\
% %		p07  & 100          & 2       & 4           & All One day  &   p23  & 168          & 4       & 6           & Mix          \\
% %		p08  & 100          & 5       & 5           & Mix               &  p24  & 51           & 6       & 3           & Mix     \\
% %		p09  & 100          & 8       & 1           & All One day  &   p25  & 51           & 6       & 3           & Mix         \\
% %		p10  & 100          & 5       & 4           & Mix              &  p26  & 51           & 6       & 3           & Mix      \\
% %		p11  & 139          & 5       & 4           & Mix              &   p27  & 102          & 6       & 6           & Mix    \\
% %		p12  & 163          & 5       & 3           & Mix               &   p28  & 102          & 6       & 6           & Mix   \\
% %		p13  & 417          & 7       & 9           & Mix               &  p29  & 102          & 6       & 6           & Mix    \\
% %		p14  & 20           & 4       & 2           & Mix           &    p30  & 153          & 6       & 9           & Mix      \\
% %		p15  & 38           & 4       & 2           & Mix          &   p31  & 153          & 6       & 9           & Mix        \\
% %		p16  & 56           & 4       & 2           & Mix         &    p32  & 153          & 6       & 9           & Mix        \\
% %		\hline
% %	\end{tabular}
% %\end{table}


% %\allowdisplaybreaks
% %\begin{table}[H]
% %	\centering
% %	\caption{Summary of PVRP instances}
% %	\label{ch3instsum}
% %	\begin{tabular}{llll|llll}
% %		\hline
% %		Name & $|\mathcal{N}|$ & $|\mathcal{D}|$ & $|\mathcal{K}|$ &  Name & $|\mathcal{N}|$ & $|\mathcal{D}|$ & $|\mathcal{K}|$ \\
% %		\hline
% %		p02  & 50           & 5       & 3                 &   p20  & 184          & 4       & 4              \\
% %	    p05  & 75           & 5       & 6                 &   p21  & 60           & 4       & 4              	 \\
% %	    p08  & 100         & 5       & 5                 &   p22  & 114          & 4       & 6                      \\
% %		p10  & 100         & 5       & 4                 &   p23  & 168          & 4       & 6                        \\
% %		p11  & 139         & 5       & 4                 &   p24  & 51           & 6       & 3                 \\
% %		p12  & 163         & 5       & 3                 &   p25  & 51           & 6       & 3              	          \\
% %		p13  & 417         & 7       & 9                 &   p26  & 51           & 6       & 3                     \\
% %	    p14  & 20           & 4       & 2                 &   p27  & 102          & 6       & 6              \\
% %	    p15  & 38           & 4       & 2                 &   p28  & 102          & 6       & 6                     \\
% %		p16  & 56           & 4       & 2                 &   p29  & 102          & 6       & 6                  \\
% %		p17  & 40           & 4       & 4                 &   p30  & 153          & 6       & 9                   \\
% %		p18  & 76           & 4       & 4                 &   p31  & 153          & 6       & 9                 \\
% %		p19  & 112         & 4       & 4                 &   p32  & 153          & 6       & 9                \\
% %		\hline
% %	\end{tabular}
% %\end{table}



% \allowdisplaybreaks
% \begin{table}[H]
% 	\centering
% 	\caption{Summary of PVRP instances}
% 	\label{ch3instsum}
% 	\begin{tabular}{lllll|lllll}
% 		\hline
% 		Name & $|\mathcal{N}|$ & $|\mathcal{D}|$ & $|\mathcal{K}|$ & BKS &  Name & $|\mathcal{N}|$ & $|\mathcal{D}|$ & $|\mathcal{K}|$ & BKS\\
% 		\hline
% 		p02  & 50          & 5       & 3   &   1322.87      &   p20  & 184          & 4       & 4        &  8367.40    \\
% 		p05  & 75          & 5       & 6    &   2024.96         &   p21  & 60           & 4       & 4     &   2170.61      	 \\
% 		p08  & 100         & 5       & 5    &  2022.47           &   p22  & 114          & 4       & 6     &   4193.45              \\
% 		p10  & 100         & 5       & 4     &  1593.43         &   p23  & 168          & 4       & 6     & 6420.71                  \\
% 		p11   & 139         & 5       & 4      &  770.89        &   p24  & 51           & 6       & 3       &   3687.46       \\
% 		p12  & 163         & 5       & 3      &  1186.47         &   p25  & 51           & 6       & 3       &   3777.15    	          \\
% 		p13  & 417         & 7       & 9       &  3492.89       &   p26  & 51           & 6       & 3        & 3795.32            \\
% 		p14  & 20           & 4       & 2     &   954.81         &   p27  & 102          & 6       & 6      &  21833.87      \\
% 		p15  & 38           & 4       & 2     &   1862.63         &   p28  & 102          & 6       & 6      &  22242.51             \\
% 		p16  & 56           & 4       & 2     &  2875.24         &   p29  & 102          & 6       & 6      &  22543.75          \\
% 		p17  & 40           & 4       & 4      & 1597.75         &   p30  & 153          & 6       & 9      &   73875.19          \\
% 		p18  & 76           & 4       & 4      & 3131.09         &   p31  & 153          & 6       & 9       &  76001.57        \\
% 		p19  & 112         & 4       & 4       &  4834.34        &   p32  & 153          & 6       & 9        &   77598.00     \\
% 		\hline
% 	\end{tabular}
% \end{table}


% All of the multi-objective algorithms considered in this paper are implemented following their descriptions in the literature.
% The same local search operators described in the previous sections are used whenever possible.
% For algorithms other than MDLS  and IMDLS, a recombination operator is required to generate offspring solution from two parent solutions.
% The crossover operator works as follows: an empty offspring solution $\pi_o$ is first created and a random recombination point $p_{cx}$ is selected from $U(1, n)$.
% Customers with identity less than $p_{cx}$ will use the same service pattern of corresponding customers in parent solution $\pi_m$, and other customers will use the same service pattern of their counterparts in parent solution $\pi_f$.
% The daily vehicle routes of the offspring solution are determined using the same procedure given in Section \ref{ch3solgen}.

% The parameter values of the multi-objective algorithms are set the same as those in \citet{imdls}.
% For MDLS and IMDLS, the number of initial solutions is set to 100 and the $F_{max}$ in IMDLS is set to 100.
% For MOEA/D, we use the weight vectors generated by  \citet{moead} for three-objective optimization problems.
% The population size is determined by the number of weight vectors and is therefore set to 351.
% For NNIA, the sizes of the dominant population, active population, and clone population are set to 100, 20, and 100, respectively.
% For SPEA2, the population size and archive size are both set to 100.
% For NSAGII, the population size is set to 100.
% For NSGAIII, the population size is determined by the size of the user-defined reference point set.
% In this paper, we use the reference point set created by \citet{nsga3} for three-objective optimization problems.
% It contains 91 reference points and the NSGAIII population size is set to 91.

% The parameter values of local search operators are set based on experience.
% For the tabu search heuristic, both $\gamma$ and $\delta$ are set to 500.0 at the beginning of the algorithm.
% The penalty updating factors $\sigma$ and $\tau$ are  set to 0.10 and 1, respectively.
% The frequency-based penalty coefficient $\lambda$ is set to 0.015 and the local search on daily routes is applied every $\xi = 50$ iterations.
% In addition, the $\gamma_d$ and $\delta_d$ values are set to corresponding $\gamma$ and $\delta$ values when the local search operator on daily routes is invoked.
% The parameter $\lambda_d$ is set to 0.015.
% All algorithms are implemented in C++ and thirty replications are solved for each instances with a time limit of two hours.

% %
% %\subsection{Metrics for comparison algorithms}
% %Three metrics are used in this paper to compare the performance of the various multi-objective algorithms for MoConPVRP.
% %
% %First, \textit{hypervolume ($I_{H}$)} is a unary operator that is able to indicate the convergence and diversity of a nondominated approximation set for the Pareto frontier \citep{coello2007}.
% %It measures the size of the objective space covered by a set of nondominated solutions $\mathcal{F}$. 
% %A reference point $z$ is necessary to compute the hypervolume of $\mathcal{F}$. 
% %For maximization problems, it is common to set z as the origin (0, 0, 0). 
% %For minimization problems, $z$ is usually set to a point with the worst values for each objective. 
% %Either way, larger hypervolumes indicate better performance. 
% %For our problem with three objectives, each solution $x \in \mathcal{F}$ in the objective space covers a cuboid defined by its coordinates ($ f_1(x), f_2(x), f_3(x)$) and the reference point $z$. 
% %The hypervolume is computed as the size of the union of all such cuboids covered by solutions in $\mathcal{F}$. 
% %The WFG algorithm descried in \citet{while2012} is used to compute the hypervolume metric. 
% %As the three objectives considered in this article do not have the same scale, it is necessary to normalize the objective values before computing the hypervolume metric.
% %To this end, for each instance, we record the best and worst values for each of the three objectives obtained by all algorithms over all 10 replications. 
% %Then all objective values are normalized into the range [0, 1] and the reference point is set to (1.1, 1.1, 1.1) for all instances.
% %
% %Second, \textit{coverage ($I_C$)} is a binary operator that compares the convergence of different nondominated solution sets \citep{zitzler2000}. 
% %It measures the extent to which one solution set $B$ is covered by another solution set $A$ by comparing the number of solutions in $B$ that are dominated by solutions in $A$ to the cardinality of $B$:
% %\begin{align}
% %I_C(A, B) = \frac{|\{b \in B: \exists \ a \in A, a \prec b\}|}{|B|}.
% %\end{align}
% %The value $I_C(A, B) = 1$ means that all solutions in $B$ are dominated by solutions in $A$,
% %while $I_C(A, B) = 0$ means that none of the solutions in $B$ are dominated by those in $A$.
% %Note that $I_C(A,B)$ does not necessarily equal $1 - I_C(B, A)$; both values need to be computed.
% %Larger values of coverage indicate better performance.
% %
% %Third, the \textit{unary multiplicative epsilon indicator} ($I_{\epsilon}$) computes the minimum factor $\epsilon$ by which each point in the reference set $R$ can be multiplied such that the resulting set is weakly dominated by set $A$ \citep{zitzler2003}:
% %\vspace{-1em}
% %\begin{align}
% %\vspace{-1em}
% %I_{\epsilon}(A, R) = \inf_{\epsilon}\{\forall r \in R\  \exists a \in A: a \preceq_{\epsilon} r \}
% %\end{align}
% %\vspace{-1em}
% %
% %This indicator is based on the $\epsilon$-dominance relation, $\preceq_{\epsilon}$, which is defined as:
% %\begin{align}
% %a \preceq_{\epsilon} r \Leftrightarrow \forall i = 1, \dots M, a_i \leq \epsilon \cdot r_i
% %\end{align}
% %for a minimization prolem with $M$ objectives and assuming that all points are positive in all objectives.
% %A smaller $I_{\epsilon}$ value indicates better performance.
% %The reference set $R$ is constructed for each instance by taking the union of all replications of all algorithms and removing dominated solutions.
% %Note that a $I_{\epsilon}$ value smaller than 1 indicates that $A$ strictly dominates the reference set $R$.

% \subsection{Algorithm performance comparison}

% To perform the comparisons, the superset is constructed for each instance-algorithm pair by taking the union of the non-dominated solution sets from all thirty replications.
% Table \ref{ch3nscount}  shows the number of non-dominated solutions in the superset corresponding to the instance and algorithm indicated by the row and column labels.
% The $\mathcal{R}$ in the last column denotes the super set  of all non-dominated solutions obtained by all multi-objective algorithms in all replications.
% The last row gives the average value across all instances.
% It can be seen from the table that NSGAII outperforms all the other comparison algorithms by finding the most number of non-dominated solutions on average.
% NNIA, SPEA2 and NSGAIII perform similarly, followed by three variations of MOEA/D.
% MDLS and IMDLS perform the worst.


% \begin{table}
% 	\centering
% 	\caption{Total number of non-dominated solutions found by each algorithm}
% 	\label{ch3nscount}
% 	\scalebox{0.9}{
% 	\begin{tabular}{lllllllllll}
% 		\hline
% 		\multirow{2}{*}{Instance} &
% 		\multirow{2}{*}{MDLS} & 
% 		\multirow{2}{*}{IMDLS} &
% 		\multicolumn{3}{c}{MOEA/D} & 
% 		\multirow{2}{*}{NNIA} & 
% 		\multirow{2}{*}{SPEA2} & 
% 		\multirow{2}{*}{NSGAII} & 
% 		\multirow{2}{*}{NSGAIII} &
% 		\multirow{2}{*}{$|\mathcal{R}|$} \\
% 		\cline{4-6}
% 		& & & WS & TCH & PBI & & & &  & \\	
% 		\hline
% 	2          &        14  &          16  &          21  &        22  &          21  &        27  &        28  &        25  &        26   & 28     \\
% 	5          &        26  &          24  &          31  &        43  &        28  &        32  &        48  &        51  &        50      & 66   \\
% 	8          &        27  &          28  &          34  &        37  &        21  &        39  &        53  &        55  &        49    &   57  \\
% 	10          &        26  &          19  &          22  &        39  &        20  &        37  &        45  &        45  &        48    &  63  \\
% 	11          &        38  &          32  &          26  &        25  &        19  &        52  &        37  &        64  &        48    &   63  \\
% 	12          &        50  &          53  &          19  &        21  &        14  &        29  &        38  &        39  &        24    &   64  \\
% 	13          &        15  &          11  &          18  &        19  &        20  &        20  &        23  &        16  &        28    &  26   \\
% 	14          &         3  &           2  &           1  &           3  &         1  &            1  &         2  &         1  &         1     &  2  \\
% 	15          &         1  &           1  &           6  &            4  &         2  &         5  &          4  &         4  &         5    &   1  \\
% 	16          &         2  &           1  &           3  &            1  &         2  &         5  &           5  &         1  &         4    &  2   \\
% 	17          &         5  &           3  &          16  &        1  4  &        15  &        25  &        24  &        19  &        14     &  24  \\
% 	18          &        16  &          28  &          24  &        31  &        13  &        35  &        39  &        63  &        52      &  26 \\
% 	19          &        16  &          20  &          25  &        26  &        22  &        36  &        36  &        52  &        25     &   33 \\
% 	20          &        24  &          25  &          35  &        24  &        32  &        50  &        55  &        28  &        43     &   26  \\
% 	21          &        28  &          17  &          16  &        15  &        15  &          21  &        16  &        34  &        39    &   27  \\
% 	22          &        14  &          22  &          24  &        23  &        33  &        44  &        39  &        44  &        52     &   43 \\
% 	23          &        18  &          31  &          33  &        23  &        29  &        45  &        43  &        44  &        34     &   49 \\
% 	24          &         7  &           7  &          28  &        19  &         31  &        22  &        24  &        32  &        30    &   35  \\
% 	25          &        11  &           9  &          22  &        24  &        21  &        28  &        25  &        26  &        29    &  37   \\
% 	26          &         6  &           5  &          17  &        17  &          15  &        15  &        20  &        20  &        27     &   23 \\
% 	27          &        22  &          26  &          45  &        36  &        32  &        57  &        44  &        51  &        48      &   59 \\
% 	28          &        22  &          34  &          24  &        34  &        26  &        55  &        44  &        43  &        43     &    66\\
% 	29          &        28  &          21  &          29  &        21  &        24  &        37  &        30  &        60  &        37     &   52 \\
% 	30          &        10  &          19  &          25  &        25  &        24  &        36  &        33  &        52  &        40     &  56  \\
% 	31          &        14  &          20  &          35  &        37  &        16  &        39  &        44  &        55  &        39     &  54  \\
% 	32          &        14  &          20  &          25  &        28  &        27  &        41  &        41  &        43  &        43     &  54  \\
% 	Average  &   17.58  &         19  &      23.23  &    23.50  &     20.11 &   32.04  &  32.31  &     \textbf{37.19}  &     33.77    &   39.85  \\
% 			\hline
% 		\end{tabular}
% }
% \end{table}



% Next, hypervolume ($I_H$), coverage ($I_C$) and unary multiplicative epsilon indicator ($I_{\epsilon}$) metrics are used in this paper to further compare the performance of the various multi-objective algorithms for MoConPVRP.
% Their definitions are detailed in \citet{imdls}.
% Table \ref{ch3hypervolume} shows the hypervolume comparison results.
% The instance number is given in the first column.
% Each of the next nine columns represents one of the comparison multi-objective algorithms implemented in this paper.
% Each entry in the table gives the hypervolume value computed based on the superset for the instance number and algorithm indicated by the row and column labels.
% The last row provides the average hypervolume value across all 26 instances for each of the nine algorithms.

% It can be seen from Table \ref{ch3hypervolume} that according to the hypervolume metric,  NSGAII performs the best among all comparison algorithms. 
% NSGAIII achieves the second best  average hypervolume value across all instances.
% All three versions of MOEA/D perform similarly, and MDLS and IMDLS perform the worst.

% \begin{table}
% 	\centering
% 	\caption{Hypervolume comparison}
% 	\label{ch3hypervolume}
% 	\scalebox{0.9}{
% 	\begin{tabular}{llllllllll}
% 		\hline
% 		\multirow{2}{*}{Instance} &
% 		\multirow{2}{*}{MDLS} & \multirow{2}{*}{IMDLS} &
% 		\multicolumn{3}{c}{MOEA/D} & \multirow{2}{*}{NNIA} & \multirow{2}{*}{SPEA2} & \multirow{2}{*}{NSGAII} & \multirow{2}{*}{NSGAIII} \\
% 		\cline{4-6}
% 		& & & WS & TCH & PBI & & & & \\	
% 		\hline
% 2       & 0.5434 & 0.5758 & 0.6523 & 0.9140 & 0.6285 & 0.6551 & 0.6628 & 0.6525 & 0.6698 \\
% 5       & 0.7903 & 0.7732 & 0.7398 & 0.9168 & 0.7424 & 0.7937 & 0.7778 & 0.9013 & 0.8084 \\
% 8       & 0.8591 & 0.7990 & 1.0468 & 1.1175 & 1.0054 & 1.1129 & 1.1056 & 1.1121 & 1.1025 \\
% 10      & 0.5245 & 0.6024 & 0.6495 & 0.6522 & 0.6331 & 0.6664 & 0.6615 & 0.6727 & 0.6778 \\
% 11      & 0.7362 & 0.7434 & 0.7811 & 0.8017 & 0.7892 & 0.8349 & 0.8201 & 0.8870 & 0.8239 \\
% 12      & 0.8032 & 0.8035 & 0.9451 & 0.9319 & 0.6927 & 0.7006 & 0.6974 & 1.0776 & 0.7004 \\
% 13      & 0.6656 & 0.6782 & 0.6015 & 0.6281 & 0.6066 & 0.6368 & 0.6244 & 0.6308 & 0.6155 \\
% 14      & 1.1875 & 1.1713 & 1.1890 & 1.1875 & 1.1890 & 1.1890 & 1.1896 & 1.1890 & 1.1890 \\
% 15      & 1.2244 & 1.2244 & 1.2148 & 1.2132 & 1.2239 & 1.2229 & 1.2222 & 1.2236 & 1.2224 \\
% 16      & 1.2165 & 1.2205 & 1.1987 & 1.2064 & 1.2053 & 1.2041 & 1.1936 & 1.2129 & 1.2093 \\
% 17      & 0.8209 & 0.7347 & 0.8273 & 0.8529 & 0.8302 & 0.8394 & 0.8373 & 0.8965 & 0.8438 \\
% 18      & 0.5930 & 0.5539 & 0.8089 & 0.7589 & 0.7799 & 0.8096 & 0.8049 & 0.7778 & 1.0170 \\
% 19      & 0.4773 & 0.4858 & 0.8021 & 0.8088 & 0.9874 & 0.9601 & 0.9839 & 1.0443 & 1.0958 \\
% 20      & 0.5522 & 0.7687 & 0.7328 & 0.7704 & 0.8089 & 0.7527 & 0.8964 & 0.9601 & 0.7737 \\
% 21      & 0.6980 & 0.7263 & 0.6688 & 0.6235 & 0.6056 & 0.8802 & 0.8819 & 0.9612 & 0.9231 \\
% 22      & 0.5568 & 0.5564 & 1.0142 & 0.5663 & 0.5627 & 1.0347 & 1.0521 & 0.9634 & 1.0349 \\
% 23      & 0.4907 & 0.5015 & 0.9522 & 0.9098 & 0.8512 & 0.9582 & 0.8642 & 1.0150 & 0.9826 \\
% 24      & 0.5765 & 0.7026 & 0.9112 & 0.8497 & 0.8807 & 0.9200 & 0.9210 & 0.9190 & 0.9416 \\
% 25      & 0.4779 & 0.4846 & 0.5939 & 0.5918 & 0.5905 & 0.5977 & 0.6158 & 0.7136 & 0.5970 \\
% 26      & 0.3873 & 0.4279 & 0.5412 & 0.5265 & 0.5368 & 0.5494 & 0.5685 & 0.6679 & 0.5693 \\
% 27      & 0.6144 & 0.5959 & 0.6475 & 0.6374 & 0.6484 & 0.6609 & 0.6632 & 0.7887 & 0.7659 \\
% 28      & 0.6034 & 0.6235 & 0.6665 & 0.7979 & 0.6555 & 0.6746 & 0.6723 & 0.8004 & 0.7985 \\
% 29      & 0.5854 & 0.6075 & 0.6663 & 0.6517 & 0.6313 & 0.6616 & 0.8585 & 0.7715 & 0.6647 \\
% 30      & 0.4967 & 0.5280 & 0.5548 & 0.5908 & 0.5399 & 0.6007 & 0.5250 & 0.7448 & 0.5876 \\
% 31      & 0.5155 & 0.5297 & 0.5739 & 0.3901 & 0.3990 & 0.5563 & 0.5379 & 0.6114 & 0.4046 \\
% 32      & 0.4495 & 0.4577 & 0.5417 & 0.3888 & 0.3827 & 0.3965 & 0.6054 & 0.5552 & 0.5508 \\
% Average & 0.6710 & 0.6876 & 0.7893 & 0.7802 & 0.7464 & 0.8027 & 0.8171 & \textbf{0.8750} & 0.8296 \\
% 		\hline
% 	\end{tabular}
% }
% \end{table}

% Table \ref{ch3coverage} shows the coverage comparison of all considered multi-objective algorithms.
% Each entry in the table indicates the percentage of solutions produced by the algorithm corresponding to the column label that are dominated by at least one solution generated by the algorithm corresponding to the row label.
% The second entry in the first row indicates that 31.49\% of the nondominated solutions generated by IMDLS are dominated by at least one solution in the nondominated solution set produced by MDLS.
% On the other hand, (as indicated by the first entry in the second row) 48.60\% of the solutions obtained by MDLS are dominated by at least one of the solutions in the set generated by IMDLS.
% It can be seen from the table that NSGAII and NSGAIII perform the best and second best among all of the comparison algorithms according to the coverage metric.
% IMDLS performs better than IMDLS on average.
% MOEA/D generally performs the worst among all algorithms.


% \begin{table}
% 	\centering
% 	\caption{Coverage comparison}
% 	\label{ch3coverage}
% 	\scalebox{0.85}{
% 	\begin{tabular}{lllllllllll}
% 	\hline
% \multirow{2}{*}{} &
% \multirow{2}{*}{MDLS} & \multirow{2}{*}{IMDLS} &
% \multicolumn{3}{c}{MOEA/D} & \multirow{2}{*}{NNIA} & \multirow{2}{*}{SPEA2} & \multirow{2}{*}{NSGAII} & \multirow{2}{*}{NSGAIII} & \multirow{2}{*}{Average}\\
% \cline{4-6}
% & & & WS & TCH & PBI & & & & & \\	
% \hline
% MDLS       & -  & 31.49\% & 35.19\% & 35.37\% & 35.62\% & 38.75\% & 37.97\% & 33.04\% & 37.69\% & 35.64\% \\
% IMDLS      & 48.60\% & -  & 39.70\% & 41.07\% & 39.04\% & 44.26\% & 44.44\% & 36.72\% & 43.14\% & 42.12\% \\
% MOEA/D-WT  & 19.93\% & 11.93\% & -  & 46.83\% & 54.19\% & 22.90\% & 25.25\% & 18.13\% & 21.52\% & 27.59\% \\
% MOEA/D-TCH & 12.57\% & 8.18\%  & 43.86\% & -  & 51.23\% & 23.77\% & 27.19\% & 19.00\% & 18.90\% & 25.59\% \\
% MOEA/D-PBI & 14.02\% & 10.81\% & 33.02\% & 38.53\% & -  & 17.20\% & 22.74\% & 14.04\% & 15.22\% & 20.70\% \\
% NNIA       & 29.52\% & 21.83\% & 59.44\% & 64.25\% & 62.47\% & -  & 47.34\% & 33.65\% & 41.28\% & 44.97\% \\
% SPEA       & 27.20\% & 21.11\% & 55.75\% & 60.25\% & 59.49\% & 39.78\% & -  & 31.51\% & 38.46\% & 41.70\% \\
% NSGAII     & 38.04\% & 27.50\% & 66.95\% & 70.18\% & 70.03\% & 50.27\% & 55.73\% & -  & 50.24\% & \textbf{53.62}\% \\
% NSGAIII    & 28.88\% & 21.58\% & 65.21\% & 67.86\% & 68.66\% & 45.74\% & 50.62\% & 33.84\% & -  & 47.80\% \\
% 		\hline
% 	\end{tabular}
% }
% \end{table}



% Table \ref{ch3epsilon} shows the unary multiplicative epsilon indicator comparisons of all multi-objective algorithms implemented in this paper.
% The first column gives the instance number and the next nine columns correspond to one of the comparison algorithms.
% Each entry in the table gives the  $I_{\epsilon}$ value computed based on the superset for the instance number and algorithm indicated by the row and column labels.
% The last row gives the average $I_{\epsilon}$ value across all 26 instances for each of the comparison algorithms.
% It can be seen from the table that SPEA2 outperforms all other multi-objective algorithms on average.
% NNIA, NSGAII and NSGAIII have similar performance with respect to this metric.

% It can be seen from the above comparisons that NSGAII and NSGAIII perform similarly among all comparison algorithms with respect to the hypervolume and coverage metrics.
% On the other hand, they are outperformed by SPEA2 with respect to the multiplicative epsilon indicator. 
% In the next section, we conduct our trade-off analysis based on the super set obtained by taking the union of all the nondominated solutions generated by all algorithms in all replications.


% \begin{table}
% 	\centering
% 	\caption{Unary multiplicative epsilon comparison}
% 	\label{ch3epsilon}
% 	\scalebox{0.90}{
% 	\begin{tabular}{llllllllll}
% 		\hline
% 	\multirow{2}{*}{Instance} &
% 	\multirow{2}{*}{MDLS} & \multirow{2}{*}{IMDLS} &
% 	\multicolumn{3}{c}{MOEA/D} & \multirow{2}{*}{NNIA} & \multirow{2}{*}{SPEA2} & \multirow{2}{*}{NSGAII} & \multirow{2}{*}{NSGAIII} \\
% 	\cline{4-6}
% 	& & & WS & TCH & PBI & & & & \\	
% 	\hline
% 2       & 19.39   & 32.72   & 11.00   & 4.93    & 11.00   & 11.00  & 11.00  & 11.00  & 11.00  \\
% 5       & 29.64   & 29.64   & 37.21   & 38.53   & 42.96   & 11.00  & 18.80  & 7.23   & 11.00  \\
% 8       & 39.57   & 37.73   & 29.14   & 23.15   & 31.86   & 11.52  & 16.19  & 10.67  & 16.14  \\
% 10      & 36.48   & 24.60   & 34.05   & 30.21   & 45.59   & 11.00  & 11.00  & 11.00  & 11.00  \\
% 11      & 21.49   & 21.04   & 20.80   & 17.86   & 19.97   & 11.00  & 11.00  & 5.27   & 11.00  \\
% 12      & 5.91    & 6.06    & 18.41   & 24.88   & 11.00   & 11.00  & 11.00  & 8.53   & 11.00  \\
% 13      & 13.37   & 11.05   & 211.88  & 154.50  & 188.63  & 126.18 & 142.19 & 135.34 & 158.10 \\
% 14      & 3.92    & 3.92    & 3.92    & 3.92    & 3.92    & 3.92   & 1.00   & 3.92   & 3.92   \\
% 15      & 1.00    & 1.00    & 6.83    & 6.83    & 6.35    & 6.35   & 6.83   & 6.35   & 6.35   \\
% 16      & 1.74    & 3.92    & 21.60   & 10.45   & 20.90   & 9.56   & 13.37  & 7.44   & 7.54   \\
% 17      & 28.98   & 63.07   & 11.00   & 11.00   & 11.00   & 11.00  & 11.00  & 5.66   & 11.00  \\
% 18      & 88.09   & 72.76   & 19.00   & 17.02   & 11.00   & 11.00  & 11.00  & 11.00  & 7.32   \\
% 19      & 206.23  & 198.27  & 17.05   & 29.83   & 34.45   & 20.69  & 13.52  & 11.53  & 3.91   \\
% 20      & 194.84  & 210.65  & 78.26   & 49.39   & 52.72   & 11.00  & 65.88  & 26.91  & 11.00  \\
% 21      & 50.79   & 22.52   & 12.51   & 17.68   & 38.77   & 3.21   & 4.40   & 2.50   & 9.94   \\
% 22      & 118.05  & 129.29  & 31.20   & 125.60  & 140.32  & 9.44   & 2.42   & 12.41  & 5.91   \\
% 23      & 178.40  & 157.95  & 359.46  & 369.52  & 288.57  & 69.62  & 69.10  & 103.97 & 95.43  \\
% 24      & 113.74  & 113.74  & 1.64    & 40.53   & 29.52   & 2.65   & 11.66  & 8.80   & 9.40   \\
% 25      & 107.89  & 114.79  & 17.36   & 38.40   & 23.81   & 14.00  & 11.00  & 8.58   & 11.00  \\
% 26      & 114.79  & 114.78  & 11.00   & 38.40   & 38.40   & 11.00  & 11.00  & 8.57   & 11.00  \\
% 27      & 520.93  & 506.58  & 58.10   & 119.80  & 123.70  & 37.30  & 36.40  & 40.20  & 53.70  \\
% 28      & 506.74  & 499.27  & 154.30  & 122.90  & 150.50  & 68.90  & 72.50  & 75.40  & 71.00  \\
% 29      & 503.30  & 503.30  & 67.50   & 75.80   & 77.60   & 30.55  & 29.98  & 2.20   & 29.98  \\
% 30      & 1692.67 & 1597.40 & 533.20  & 127.80  & 1231.70 & 441.60 & 352.90 & 521.60 & 388.30 \\
% 31      & 1520.05 & 1491.95 & 1021.40 & 1326.90 & 1285.60 & 332.90 & 368.50 & 415.70 & 378.10 \\
% 32      & 1467.57 & 1439.47 & 277.40  & 594.20  & 1233.70 & 89.90  & 37.70  & 200.70 & 137.80 \\
% Average & 291.75  & 284.90  & 117.89  & 131.54  & 198.21  & 52.97  & \textbf{51.97}  & 63.94  & 56.99  \\
% 		\hline
% 	\end{tabular}
% }
% \end{table}

% \subsection{Trade-off analysis}
% In this section, we conduct the trade-off analysis using the super set $\mathcal{R}$ of all non-dominated solutions obtained by all multi-objective algorithms in all replications.
% Table \ref{ch3superstat} gives some statistics on each of the three objectives for solutions in  $\mathcal{R}$.
%  Specifically, the minimal value, maximal value, average value and standard deviation of each objective are given.
% Additionally, the cardinality of $\mathcal{R}$ is given in the second column. 
 
%  \begin{table}[H]
%  	\centering
%  	\caption{Statistics on superset solutions}
%  	\label{ch3superstat}
%  	\scalebox{0.85}{
%  		\begin{tabular}{llllllllllllll}
%  			\hline
%  			\multirow{2}{*}{Inst.}  & \multirow{2}{*}{$|\mathcal{R}|$} & \multicolumn{4}{c|}{$f_{TD}$} & \multicolumn{4}{c|}{$f_{DC}$} & \multicolumn{4}{c}{$f_{TC}$} \\
%  			\cline{3-14}
%  			& & \multicolumn{1}{c}{min} & \multicolumn{1}{c}{max} & \multicolumn{1}{c}{avg} & \multicolumn{1}{c|}{std} & \multicolumn{1}{c}{min} & \multicolumn{1}{c}{max} & \multicolumn{1}{c}{avg} & \multicolumn{1}{c|}{std} & \multicolumn{1}{c}{min}  & \multicolumn{1}{c}{max} & \multicolumn{1}{c}{avg} & \multicolumn{1}{c}{std}\\
%  			\hline
%  			  2  &  28  &     1322.87  &  1496.99  &  1360.44  &  42.31  &   1.00  &   3.00  &   2.54  &   0.57  &  11.02  &  69.37  &  38.70  &  18.84  \\  
%  			 5  &   66 &    2033.39  &  2633.29  &  2184.98  &  132.48  &   2.00  &   5.00  &   3.58  &   0.85  &  14.44  &  100.47  &  44.43  &  19.72  \\  
%  			 8  &   57 &    2030.67  &  2646.78  &  2149.92  &  112.97  &   2.00  &   5.00  &   3.12  &   0.92  &  17.77  &  106.89  &  58.88  &  23.22  \\  
%  			 10  &  63  &    1601.51  &  2340.17  &  1749.72  &  162.86  &   1.00  &   3.00  &   2.46  &   0.53  &  11.95  &  110.97  &  44.47  &  28.19  \\  
%  			 11  &  63 &    778.19  &  1202.00  &  863.69  &  110.24  &   1.00  &   4.00  &   2.81  &   0.75  &   4.21  &  80.94  &  26.64  &  15.31  \\  
%  			 12  &  64  &    1197.59  &  1548.25  &  1262.68  &  89.62  &   1.00  &   3.00  &   2.36  &   0.51  &   1.24  &  65.70  &  19.58  &  16.21  \\  
%  			 13  &  26  &    3649.46  &  4461.83  &  3916.80  &  246.37  &   2.00  &   2.00  &   2.00  &   0.00  &   1.80  &  47.63  &  13.95  &  10.95  \\  
%  			 14  &  2  &    954.81  &  1122.42  &  1038.61  &  83.81  &   1.00  &   2.00  &   1.50  &   0.50  &   0.00  &   2.92  &   1.46  &   1.46  \\  
%  			 15  &   1 &    1862.63  &  1862.63  &  1862.63  &   0.00  &   1.00  &   1.00  &   1.00  &   0.00  &  12.99  &  12.99  &  12.99  &   0.00  \\  
%  			 16  &   2 &    2875.24  &  2875.24  &  2875.24  &   0.00  &   1.00  &   2.00  &   1.50  &   0.50  &  20.51  &  23.43  &  21.97  &   1.46  \\  
%  			 17  &   24 &    1597.75  &  2062.33  &  1739.48  &  117.27  &   1.00  &   4.00  &   2.54  &   0.87  &   1.79  &  131.98  &  35.94  &  38.16  \\  
%  			 18  &   26 &    3131.09  &  3725.92  &  3231.85  &  130.30  &   1.00  &   4.00  &   2.69  &   0.82  &   8.58  &  233.89  &  110.84  &  50.40  \\  
%  			 19  &   33 &    4834.34  &  5741.25  &  4940.81  &  191.46  &   1.00  &   4.00  &   2.36  &   0.81  &  48.94  &  379.50  &  243.73  &  65.11  \\  
%  			 20  &   26 &    8367.40  &  9195.89  &  8540.67  &  217.80  &   1.00  &   4.00  &   2.38  &   0.79  &  98.48  &  586.59  &  405.88  &  127.40  \\  
%  			 21  &  27 &    2171.45  &  2732.05  &  2300.18  &  174.46  &   2.00  &   4.00  &   3.19  &   0.77  &   2.64  &  130.00  &  55.64  &  38.16  \\  
%  			 22  &  43  &    4194.88  &  5382.85  &  4616.69  &  352.87  &   2.00  &   4.00  &   3.16  &   0.74  &  40.84  &  236.83  &  116.44  &  49.13  \\  
%  			 23  & 49  &    6420.71  &  9018.69  &  7119.17  &  579.09  &   2.00  &   4.00  &   3.00  &   0.78  &  86.35  &  299.62  &  212.01  &  67.65  \\  
%  			 24  & 35   &    3687.46  &  5047.45  &  4190.56  &  353.26  &   2.00  &   3.00  &   2.63  &   0.48  &   5.02  &  166.69  &  69.44  &  42.25  \\  
%  			 25  & 37   &    3777.15  &  4908.85  &  4157.61  &  308.32  &   1.00  &   3.00  &   2.59  &   0.54  &   3.97  &  174.35  &  75.31  &  44.10  \\  
%  			 26  & 23  &    3795.32  &  4742.01  &  4179.15  &  262.59  &   1.00  &   3.00  &   2.43  &   0.58  &   3.97  &  119.44  &  61.14  &  33.63  \\  
%  			 27  &  59 &    21939.10  &  30268.70  &  24932.05  &  2852.84  &   2.00  &   5.00  &   4.07  &   1.12  &  90.51  &  1082.53  &  598.61  &  336.41  \\  
%  			 28  &  66  &    22287.20  &  31416.90  &  25034.13  &  2856.97  &   2.00  &   6.00  &   4.23  &   1.13  &  90.51  &  1058.72  &  502.42  &  298.41  \\  
%  			 29  &  52  &    22594.10  &  31548.50  &  24678.04  &  2942.13  &   2.00  &   6.00  &   4.29  &   1.06  &  86.49  &  1092.28  &  549.73  &  277.19  \\  
%  			 30  &  56  &    74663.70  &  111507.00  &  90092.45  &  11315.34  &   2.00  &   6.00  &   4.70  &   1.31  &  219.49  &  3265.53  &  1573.54  &  994.70  \\  
%  			 31  &   54&    76888.80  &  116247.00  &  93744.38  &  13459.82  &   2.00  &   6.00  &   4.33  &   1.61  &  275.40  &  3328.40  &  1858.95  &  990.03  \\  
%  			 32  &  54 &    78504.00  &  118004.00  &  91356.89  &  12226.91  &   2.00  &   6.00  &   4.83  &   1.24  &  327.88  &  3116.58  &  1556.07  &  870.78  \\
%  			\hline 
%  		\end{tabular}
%  	}
%  \end{table}



% There usually exist many nondominated solutions in $\mathcal{R}$ and it is desirable to identify the best compromise solution to facilitate managerial decision making.
% Specifically, the level diagram technique proposed by \citet{leveldiagram} is employed here to identify four special solutions in the super set of each instance.
% First, the objective values of non-dominated solutions in $\mathcal{R}$ are normalized on a [0,1] scale. 
% Next, the euclidean norm is computed for each solution $s \in \mathcal{R}$, where $\bar{f_i}$ represents the normalized value of objective $i$:
% $
% E(s) = \sqrt{\sum_{i = 1}^3 \bar{f_i}^2(s)}.
% $
% Let $s^{BC}$ denote the best compromise solution in the super set $\mathcal{R}$.
% Furthermore, let $s^{TD}$, $s^{DC}$ and $s^{TC}$ denote the solutions with best travel distance, driver consistency, and time consistency value, respectively.
% Then $s^{BC}$ is the solution in $\mathcal{R}$ that has the smallest $E(s)$ value.
% Similarly, $s^{TD}$, $s^{DC}$ and $s^{TC}$ can be identified from $\mathcal{R}$ with the corresponding minimal objective values.
% Note that there may exist multiple solutions with minimal driver consistency value, in this case, the solution with the smallest $E(s)$ is chosen as $s^{DC}$.









% Table \ref{ch3foursols} shows these four solutions for each of the 26 instances.
% For each objective value of $s^{BC}$, $s^{DC}$ and $s^{TC}$, its percentage difference with the $s^{TD}$ is also included in the parenthesis.
% Take the best compromise solution $s^{BC}$ and the best travel distance solution $s^{TD}$ for instance 5 in Table \ref{ch3foursols} as an example.
% The travel distance of $s^{TD}$ for this instance is 2033.39, with driver and time consistency values being 4 and 71.83, respectively.
% The travel distance of the best compromise solution $s^{BC}$ is 2194.68, approximately 7.93\% higher than the travel distance in the best travel distance solution.
% However, the increase in travel distance results in better consistency values, with driver and time consistencies of 3 (a 25.00\% decrease) and 36.93 ( a 48.59\% decrease), respectively.
% To validate the quality of the superset $\mathcal{R}$, table \ref{ch3tdbkscomp} presents for each instance the travel distance comparison between the best known solution and $s^{TD}$.
% It can be seen from the table that for most instances the gap is within 1\%.

% %
% %\begin{table}
% %	\centering
% %	\caption{Four solutions}
% %	\label{ch3foursols}
% %	\scalebox{0.8}{
% %	\begin{tabular}{lllllllllllll}
% %		\hline
% %		 \multirow{2}{*}{Inst.} & \multicolumn{3}{c|}{$s^{BC}$} & \multicolumn{3}{c|}{$s^{TD}$} & \multicolumn{3}{c|}{$s^{DC}$} & \multicolumn{3}{c}{$s^{TC}$}\\
% %		 \cline{2 - 13}
% %		& \multicolumn{1}{c}{$f_{TD}$} & \multicolumn{1}{c}{$f_{DC}$} & \multicolumn{1}{c|}{$f_{TC}$} & \multicolumn{1}{c}{$f_{TD}$} & \multicolumn{1}{c}{$f_{DC}$} & \multicolumn{1}{c|}{$f_{TC}$} & \multicolumn{1}{c}{$f_{TD}$} & \multicolumn{1}{c}{$f_{DC}$} & \multicolumn{1}{c|}{$f_{TC}$}  & \multicolumn{1}{c}{$f_{TD}$} & \multicolumn{1}{c}{$f_{DC}$} & \multicolumn{1}{c}{$f_{TC}$}\\
% %		\hline
% %		2  & 1369.46 & 2 & 21.03   & 1322.87 & 2 & 69.37   & 1427.94 & 1 & 60.11   & 1455.23 & 3 & 11.02  \\
% %		5  & 2194.68 & 3 & 36.93   & 2033.39 & 4 & 71.83   & 2380.15 & 2 & 68.50   & 2440.39 & 5 & 14.44  \\
% %		8  & 2229.49 & 2 & 37.22   & 2030.67 & 4 & 92.46   & 2229.49 & 2 & 37.22   & 2441.48 & 4 & 17.77  \\
% %		10 & 1745.45 & 2 & 28.64   & 1601.51 & 3 & 110.97  & 2340.17 & 1 & 110.23  & 2194.7  & 3 & 11.95  \\
% %		11 & 828.076 & 2 & 23.76   & 778.188 & 3 & 57.92   & 1113.9  & 1 & 80.94   & 1202    & 3 & 4.21   \\
% %		12 & 1227.33 & 2 & 9.70    & 1197.59 & 3 & 65.70   & 1279.49 & 1 & 37.79   & 1478.3  & 3 & 1.24   \\
% %		13 & 3737.55 & 2 & 12.06   & 3649.46 & 2 & 47.63   & 3737.55 & 2 & 12.06   & 4461.83 & 2 & 1.80   \\
% %		14 & 954.807 & 1 & 2.92    & 954.807 & 1 & 2.92    & 954.807 & 1 & 2.92    & 1122.42 & 2 & 0.00   \\
% %		15 & 1862.63 & 1 & 12.99   & 1862.63 & 1 & 12.99   & 1862.63 & 1 & 12.99   & 1862.63 & 1 & 12.99  \\
% %		16 & 2875.24 & 1 & 23.43   & 2875.24 & 1 & 23.43   & 2875.24 & 1 & 23.43   & 2875.24 & 2 & 20.51  \\
% %		17 & 1754.28 & 2 & 7.51    & 1597.75 & 2 & 63.86   & 2062.33 & 1 & 29.77   & 1805.69 & 3 & 1.79   \\
% %		18 & 3302.9  & 2 & 65.66   & 3131.09 & 3 & 137.44  & 3246.29 & 1 & 190.32  & 3303.61 & 3 & 8.58   \\
% %		19 & 5029.81 & 2 & 73.90   & 4834.34 & 3 & 379.50  & 4866.27 & 1 & 291.70  & 4926.8  & 3 & 48.94  \\
% %		20 & 8713.2  & 3 & 98.48   & 8367.4  & 2 & 523.90  & 8564.24 & 1 & 556.76  & 8713.2  & 3 & 98.48  \\
% %		21 & 2189.93 & 3 & 22.52   & 2171.45 & 3 & 116.70  & 2234.07 & 2 & 99.15   & 2536.09 & 4 & 2.64   \\
% %		22 & 4883.79 & 2 & 92.14   & 4194.88 & 3 & 185.07  & 4883.79 & 2 & 92.14   & 5123.74 & 4 & 40.84  \\
% %		23 & 7656.99 & 2 & 154.76  & 6420.71 & 3 & 297.29  & 7656.99 & 2 & 154.76  & 8445.81 & 4 & 86.35  \\
% %		24 & 4152.39 & 2 & 61.71   & 3687.46 & 2 & 166.69  & 4152.39 & 2 & 61.71   & 5047.45 & 3 & 5.02   \\
% %		25 & 4196.99 & 2 & 61.71   & 3777.15 & 3 & 171.65  & 4908.85 & 1 & 103.31  & 4781.29 & 3 & 3.97   \\
% %		26 & 4165.78 & 2 & 55.45   & 3795.32 & 2 & 119.44  & 4742.01 & 1 & 68.34   & 4585.38 & 3 & 3.97   \\
% %		27 & 23515.2 & 4 & 283.25  & 21939.1 & 5 & 1050.67 & 28651.9 & 2 & 1014.47 & 30268.7 & 5 & 90.51  \\
% %		28 & 23882.1 & 4 & 281.78  & 22287.2 & 5 & 990.18  & 29808.4 & 2 & 994.57  & 27888.9 & 5 & 90.51  \\
% %		29 & 24078.7 & 4 & 280.26  & 22594.1 & 5 & 985.63  & 31548.5 & 2 & 668.91  & 29114   & 6 & 86.49  \\
% %		30 & 83237.8 & 4 & 1766.35 & 74663.7 & 6 & 2921.05 & 108334  & 2 & 2421.25 & 103914  & 5 & 219.49 \\
% %		31 & 90974   & 4 & 1766.35 & 76888.8 & 6 & 2498.75 & 111162  & 2 & 2705.94 & 100679  & 6 & 275.40 \\
% %		32 & 95295.3 & 4 & 1323.67 & 78504   & 6 & 2903.90 & 109916  & 2 & 2935.61 & 101836  & 6 & 327.88 \\
% %		\hline 
% %	\end{tabular}
% %}
% %\end{table}




% \begin{table}
% 	\centering
% 	\caption{Four solutions}
% 	\label{ch3foursols}
% 	\scalebox{0.5}{
% 		\begin{tabular}{lllllllllllll}
% 			\hline
% 			\multirow{2}{*}{Inst.} & \multicolumn{3}{c|}{$s^{BC}$} & \multicolumn{3}{c|}{$s^{TD}$} & \multicolumn{3}{c|}{$s^{DC}$} & \multicolumn{3}{c}{$s^{TC}$}\\
% 			\cline{2 - 13}
% 			& \multicolumn{1}{c}{$f_{TD}$} & \multicolumn{1}{c}{$f_{DC}$} & \multicolumn{1}{c|}{$f_{TC}$} & \multicolumn{1}{c}{$f_{TD}$} & \multicolumn{1}{c}{$f_{DC}$} & \multicolumn{1}{c|}{$f_{TC}$} & \multicolumn{1}{c}{$f_{TD}$} & \multicolumn{1}{c}{$f_{DC}$} & \multicolumn{1}{c|}{$f_{TC}$}  & \multicolumn{1}{c}{$f_{TD}$} & \multicolumn{1}{c}{$f_{DC}$} & \multicolumn{1}{c}{$f_{TC}$}\\
% 			\hline
% 	  2  &    1369.460(  3.52\%)  &           2(  0.00\%)  &       21.03( 69.69\%)  &         1322.870  &      2  &       69.37  &         1427.940( 7.94\%)  &      1(50.00\%)  &       60.11( 13.34\%)  &         1455.230( 10.01\%)  &      3( 50.00\%)  &       11.02( 84.11\%) \\
% 	5  &    2194.680(  7.93\%)  &           3( 25.00\%)  &       36.93( 48.59\%)  &         2033.390  &      4  &       71.83  &         2380.150(17.05\%)  &      2(50.00\%)  &       68.50(  4.63\%)  &         2440.390( 20.02\%)  &      5( 25.00\%)  &       14.44( 79.90\%) \\
% 	8  &    2229.490(  9.79\%)  &           2( 50.00\%)  &       37.22( 59.74\%)  &         2030.670  &      4  &       92.45  &         2229.490( 9.79\%)  &      2(50.00\%)  &       37.22( 59.74\%)  &         2441.480( 20.23\%)  &      4( -0.00\%)  &       17.77( 80.78\%) \\
% 	10  &    1745.450(  8.99\%)  &           2( 33.33\%)  &       28.64( 74.19\%)  &         1601.510  &      3  &      110.97  &         2340.170(46.12\%)  &      1(66.67\%)  &      110.23(  0.66\%)  &         2194.700( 37.04\%)  &      3( -0.00\%)  &       11.95( 89.23\%) \\
% 	11  &     828.076(  6.41\%)  &           2( 33.33\%)  &       23.76( 58.98\%)  &          778.188  &      3  &       57.92  &         1113.900(43.14\%)  &      1(66.67\%)  &       80.94(-39.75\%)  &         1202.000( 54.46\%)  &      3( -0.00\%)  &        4.21( 92.73\%) \\
% 	12  &    1227.330(  2.48\%)  &           2( 33.33\%)  &        9.70( 85.23\%)  &         1197.590  &      3  &       65.70  &         1279.490( 6.84\%)  &      1(66.67\%)  &       37.79( 42.48\%)  &         1478.300( 23.44\%)  &      3( -0.00\%)  &        1.24( 98.11\%) \\
% 	13  &    3737.550(  2.41\%)  &           2(  0.00\%)  &       12.06( 74.68\%)  &         3649.460  &      2  &       47.63  &         3737.550( 2.41\%)  &      2( 0.00\%)  &       12.06( 74.68\%)  &         4461.830( 22.26\%)  &      2( -0.00\%)  &        1.80( 96.22\%) \\
% 	14  &     954.807(  0.00\%)  &           1(  0.00\%)  &        2.92(  0.00\%)  &          954.807  &      1  &        2.92  &          954.807( 0.00\%)  &      1( 0.00\%)  &        2.92(  0.00\%)  &         1122.420( 17.55\%)  &      2(100.00\%)  &        0.00(100.00\%) \\
% 	15  &    1862.630(  0.00\%)  &           1(  0.00\%)  &       12.99(  0.00\%)  &         1862.630  &      1  &       12.99  &         1862.630( 0.00\%)  &      1( 0.00\%)  &       12.99(  0.00\%)  &         1862.630(  0.00\%)  &      1( -0.00\%)  &       12.99(  0.00\%) \\
% 	16  &    2875.240(  0.00\%)  &           1(  0.00\%)  &       23.43(  0.00\%)  &         2875.240  &      1  &       23.43  &         2875.240( 0.00\%)  &      1( 0.00\%)  &       23.43(  0.00\%)  &         2875.240(  0.00\%)  &      2(100.00\%)  &       20.51( 12.44\%) \\
% 	17  &    1754.280(  9.80\%)  &           2(  0.00\%)  &        7.51( 88.24\%)  &         1597.750  &      2  &       63.86  &         2062.330(29.08\%)  &      1(50.00\%)  &       29.77( 53.38\%)  &         1805.690( 13.01\%)  &      3( 50.00\%)  &        1.79( 97.19\%) \\
% 	18  &    3302.900(  5.49\%)  &           2( 33.33\%)  &       65.66( 52.23\%)  &         3131.090  &      3  &      137.44  &         3246.290( 3.68\%)  &      1(66.67\%)  &      190.32(-38.47\%)  &         3303.610(  5.51\%)  &      3( -0.00\%)  &        8.58( 93.76\%) \\
% 	19  &    5029.810(  4.04\%)  &           2( 33.33\%)  &       73.90( 80.53\%)  &         4834.340  &      3  &      379.50  &         4866.270( 0.66\%)  &      1(66.67\%)  &      291.69( 23.14\%)  &         4926.800(  1.91\%)  &      3( -0.00\%)  &       48.94( 87.10\%) \\
% 	20  &    8713.200(  4.13\%)  &           3(-50.00\%)  &       98.48( 81.20\%)  &         8367.400  &      2  &      523.89  &         8564.240( 2.35\%)  &      1(50.00\%)  &      556.76( -6.27\%)  &         8713.200(  4.13\%)  &      3( 50.00\%)  &       98.48( 81.20\%) \\
% 	21  &    2189.930(  0.85\%)  &           3(  0.00\%)  &       22.52( 80.71\%)  &         2171.450  &      3  &      116.70  &         2234.070( 2.88\%)  &      2(33.33\%)  &       99.15( 15.04\%)  &         2536.090( 16.79\%)  &      4( 33.33\%)  &        2.64( 97.74\%) \\
% 	22  &    4883.790( 16.42\%)  &           2( 33.33\%)  &       92.14( 50.21\%)  &         4194.880  &      3  &      185.07  &         4883.790(16.42\%)  &      2(33.33\%)  &       92.14( 50.21\%)  &         5123.740( 22.14\%)  &      4( 33.33\%)  &       40.84( 77.93\%) \\
% 	23  &    7656.990( 19.25\%)  &           2( 33.33\%)  &      154.76( 47.94\%)  &         6420.710  &      3  &      297.29  &         7656.990(19.25\%)  &      2(33.33\%)  &      154.76( 47.94\%)  &         8445.810( 31.54\%)  &      4( 33.33\%)  &       86.35( 70.95\%) \\
% 	24  &    4152.390( 12.61\%)  &           2(  0.00\%)  &       61.71( 62.98\%)  &         3687.460  &      2  &      166.69  &         4152.390(12.61\%)  &      2( 0.00\%)  &       61.71( 62.98\%)  &         5047.450( 36.88\%)  &      3( 50.00\%)  &        5.02( 96.99\%) \\
% 	25  &    4196.990( 11.12\%)  &           2( 33.33\%)  &       61.71( 64.05\%)  &         3777.150  &      3  &      171.65  &         4908.850(29.96\%)  &      1(66.67\%)  &      103.31( 39.81\%)  &         4781.290( 26.58\%)  &      3( -0.00\%)  &        3.97( 97.69\%) \\
% 	26  &    4165.780(  9.76\%)  &           2(  0.00\%)  &       55.45( 53.58\%)  &         3795.320  &      2  &      119.44  &         4742.010(24.94\%)  &      1(50.00\%)  &       68.34( 42.78\%)  &         4585.380( 20.82\%)  &      3( 50.00\%)  &        3.97( 96.67\%) \\
% 	27  &   23515.200(  7.18\%)  &           4( 20.00\%)  &      283.25( 73.04\%)  &        21939.100  &      5  &     1050.67  &        28651.900(30.60\%)  &      2(60.00\%)  &     1014.47(  3.45\%)  &        30268.700( 37.97\%)  &      5( -0.00\%)  &       90.51( 91.39\%) \\
% 	28  &   23882.100(  7.16\%)  &           4( 20.00\%)  &      281.78( 71.54\%)  &        22287.200  &      5  &      990.18  &        29808.400(33.75\%)  &      2(60.00\%)  &      994.57( -0.44\%)  &        27888.900( 25.13\%)  &      5( -0.00\%)  &       90.51( 90.86\%) \\
% 	29  &   24078.700(  6.57\%)  &           4( 20.00\%)  &      280.25( 71.57\%)  &        22594.100  &      5  &      985.63  &        31548.500(39.63\%)  &      2(60.00\%)  &      668.91( 32.13\%)  &        29114.000( 28.86\%)  &      6( 20.00\%)  &       86.49( 91.23\%) \\
% 	30  &   83237.800( 11.48\%)  &           4( 33.33\%)  &     1766.35( 39.53\%)  &        74663.700  &      6  &     2921.05  &       108334.000(45.10\%)  &      2(66.67\%)  &     2421.25( 17.11\%)  &       103914.000( 39.18\%)  &      5(-16.67\%)  &      219.49( 92.49\%) \\
% 	31  &   90974.000( 18.32\%)  &           4( 33.33\%)  &     1766.35( 29.31\%)  &        76888.800  &      6  &     2498.75  &       111162.000(44.58\%)  &      2(66.67\%)  &     2705.94( -8.29\%)  &       100679.000( 30.94\%)  &      6( -0.00\%)  &      275.40( 88.98\%) \\
% 	32  &   95295.300( 21.39\%)  &           4( 33.33\%)  &     1323.67( 54.42\%)  &        78504.000  &      6  &     2903.90  &       109916.000(40.01\%)  &      2(66.67\%)  &     2935.61( -1.09\%)  &       101836.000( 29.72\%)  &      6( -0.00\%)  &      327.88( 88.71\%) \\
	
% 			\hline 
% 		\end{tabular}
% 	}
% \end{table}






% \allowdisplaybreaks
% \begin{table}[H]
% 	\centering
% 	\caption{Travel distance comparison between $s^{TD}$ and BKS}
% 	\label{ch3tdbkscomp}
% 	\begin{tabular}{llll|llll}
% 		\hline
% 		Name &  BKS & $s^{TD}$ & Gap  &  Name & BKS & $s^{TD}$ & Gap \\
% 		\hline
% 		p2  & 1322.87 & 1322.87 & 0.00\% & p20 & 8367.4   & 8367.4  & 0.00\% \\
% 		p5  & 2024.96 & 2033.39 & 0.42\% & p21 & 2170.61  & 2171.45 & 0.04\% \\
% 		p8  & 2022.47 & 2030.67 & 0.41\% & p22 & 4193.45  & 4194.88 & 0.03\% \\
% 		p10 & 1593.43 & 1601.51 & 0.51\% & p23 & 6420.71  & 6420.71 & 0.00\% \\
% 		p11 & 770.89  & 778.188 & 0.95\% & p24 & 3687.46  & 3687.46 & 0.00\% \\
% 		p12 & 1186.47 & 1197.59 & 0.94\% & p25 & 3777.15  & 3777.15 & 0.00\% \\
% 		p13 & 3492.89 & 3649.46 & 4.48\% & p26 & 3795.32  & 3795.32 & 0.00\% \\
% 		p14 & 954.81  & 954.807 & 0.00\% & p27 & 21833.87 & 21939.1 & 0.48\% \\
% 		p15 & 1862.63 & 1862.63 & 0.00\% & p28 & 22242.51 & 22287.2 & 0.20\% \\
% 		p16 & 2875.24 & 2875.24 & 0.00\% & p29 & 22543.75 & 22594.1 & 0.22\% \\
% 		p17 & 1597.75 & 1597.75 & 0.00\% & p30 & 73875.19 & 74663.7 & 1.07\% \\
% 		p18 & 3131.09 & 3131.09 & 0.00\% & p31 & 76001.57 & 76888.8 & 1.17\% \\
% 		p19 & 4834.34 & 4834.34 & 0.00\% & p32 & 77598    & 78504   & 1.17\%\\
% 		\hline
% 	\end{tabular}
% \end{table}




% Table \ref{ch3summary} summarizes the trade-offs between the objectives of travel cost minimization and service consistency maximization.
% Specifically, three pairs of solutions are compared: 
% the best travel distance solution compared with the best compromise solution, the best travel distance compared with the best driver consistency solution, and the best travel distance compared with the best time consistency solution.  
% The first row indicates that the best compromise solution increases travel distance by  7.97\% on average and decreases driver and time consistency by 17.37\% and 56.62\% on average, respectively, when compared with the solution obtained by optimizing travel distance alone. That is, an approximate 17\% improvement in driver consistency and 56\% improvement in time consistency comes at a cost of a 8\% increase in travel distance. 
% The second row shows that, if only driver consistency is considered,  the best driver consistency solution, on average, increases travel distance  by  19.57\%, decreases the driver consistency  by 45.38\%, and time consistency by 18.82\%, when compared with the solution with best travel distance.  
% It can be seen from the table that $f_{TC}$ improves when optimizing driver consistency alone but $f_{DC}$ worsens when optimizing time consistency alone, which contradicts the observations made in the case of multi-objective ConVRP \citep{imdls}.
% This can be explained by the fact that the driver consistency optimizer tends to use fewer vehicles to visit a customer and that will increase the possibility of a vehicle visiting the similar set of customers across the planning horizon. 
% With the similar set of customers, the order in which a vehicle visits those customer should be similar as well, which indirectly helps the time consistency objective.
% On the other hand, the time consistency optimizer is solely focused on improving time consistency objective without potential  benefits to the driver consistency objective.
% The driver consistency of a customer will be improved if the time consistency algorithm happens to use fewer number of vehicles to visit the customer, in other cases, the driver consistency of the customer will be negatively impacted. 
% Therefore, the overall driver consistency objective change under the impact of time consistency optimizer will vary case by case, as can be seen from the case of MoConVRP and MoConPVRP.





% 	\begin{table}
% 	\centering
% 	%\scalebox{1.0}{
% 	\begin{tabular}{crrr}
% 		\hline
% 		& \multicolumn{1}{c}{$f_{TD}$} & \multicolumn{1}{c}{$f_{DC}$} & \multicolumn{1}{c}{$f_{TC}$}\\
% 		\hline
% 		$s^{TD}$ vs. $s^{BC}$   & 7.97\% \ding{218} & 17.37\% \ding{216}& 56.62\% \ding{216}\\
% 		$s^{TD}$ vs.  $s^{DC}$   &19.57\% \ding{218}  & 45.38\% \ding{216} &  18.82\% \ding{216}\\
% 		$s^{TD}$ vs. $s^{TC}$   & 22.16\% \ding{218} & 22.24\% \ding{218} & 86.63\% \ding{216}\\
% 		\hline
% 	\end{tabular}
% 	%}
% 	\caption{Comparisons of four solutions}
% 	\label{ch3summary}
% \end{table}



% \section{Conclusion}\label{conclusion}
% This paper studies service consistency in the context of periodic vehicle routing problems using a multi-objective optimization approach and defines, for the first time in the literature, the MoConPVRP.
% Two service consistency objectives (maximization of driver consistency and maximization of time consistency) are considered separately with the traditional objective of minimizing total travel distance.
% Seven multi-objective optimization algorithms are employed to solve the studied problem and their performance is validated on a total of 26  benchmark instances taken from the literature.
% Trade-off analysis on the nondominated solutions obtained by all algorithms suggest that pursuing the best compromise solution among all three objectives may increase travel cost by about 8\% while improving driver and time consistency by approximately 17\% and 56\% on average, when compared with a compromise solution having lowest overall travel distance.
% Directions for future work can include studying service consistency in PVRP using a single objective approach where travel distance is minimized while service consistency is enforced via hard constraints.




%
%\newpage
%\singlespacing
%\bibliographystyle{plainnat}
%%\bibliography{bibliopvrp}
%\bibliography{mybiblio}
%

\bibliographystyle{plainnat}

\bibliography{biblio}



\end{document}
