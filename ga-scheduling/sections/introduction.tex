Jackson first proposed the concept of dynamic scheduling in 1957 considering the difficulty of applying traditional static scheduling models in real-world manufacturing environments that are characterized by dynamism, randomness and multiple objectives. 
The proposal of dynamic scheduling models is mainly due to frequent occurrences of disturbed schedules inflicted by a series of stochastic impacting factors encompassing new order arrivals, part scrap and rework, due date changes, machine malfunction or delayed raw material arrivals.
Rescheduling is therefore necessary in order to resume manufacturing process based on the updated part status in the system.

Traditional integer programming methods can be hardly used in solving real-world dynamic scheduling problems.
On the other hand, advancements in computer technologies open new opportunities in solving dynamic scheduling problems using artificial intelligence, simulation, rolling-horizon rescheduling and meta-heuristics, which lays the foundation for practical usage of manufacturing scheduling models.
Artificial intelligence and expert systems can identify the best scheduling strategy by searching knowledge databases based on current system status and predefined optimization objective.
However, they suffer from poor adaptivity to new environment, high development cost and prolonged development cycle.
Discrete system simulation method simulates real-world manufacturing environment by creating simulation models, but general principles are difficult to find due to its experimental nature.
Rolling horizon rescheduling was original proposed by Nelson in 1977 and has received numerous research attentions and applications in recent years.
It divides the dynamic scheduling process into multiple continuous scheduling horizons and then conducts online optimization for each horizon in order to find the individual optimal solution, which makes it applicable to complex dynamic manufacturing scheduling environments \citep{bierwirth19991, jian19972}.

Genetic algorithm, one of the meta-heuristic algorithms, has witnessed many applications in dynamic scheduling researches due to its simple operations, high efficiency, good robustness, superior adaptability and intriguing searching capability based on individual fitness. 
This paper proposes a hybrid algorithmic framework of multi-objective genetic algorithm and rolling horizon rescheduling to solve the real-world dynamic scheduling problem considering possible delays of raw material arrival, part machining times and assembly times.
It can also tackle the emergent insertion of parts and continuous arrivals of planned jobs.
This framework can be used in other manufacturing scenarios as well.



















