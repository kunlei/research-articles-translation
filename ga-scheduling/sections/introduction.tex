
Traditional static scheduling models cannot be used in real-world manufacturing environment with dynamic, stochastic and multi-objective nature, and the concept of dynamic scheduling was first proposed by Jackson in 1957.
The advent of dynamic scheduling models is due to the disturbed schedules inflicted by a series of stochastic impacting factors emcompassing the arrival of new orders, scrap and rework of certain parts, changed due date, machine malfunction or delayed raw material arrival.
Rescheduling is therefore necessary in order to resume manufacturing process based on the current part status in the system.

Traditional integer programming methods can be hardly used in solving real-world dynamic scheduling problems.
On the other hand, new advancemens in computer technologies open new opportunities in sovling dynamic scheduling problems using artificial intelligence, simulation, rolling-horizon rescheduling and metaheuristics, which lays the foundations for practical usage of manufacturing scheduling models.
Artificial intelligence and expert systems can identify the best schedulig strategy by searching knowledge database based on current system status and predefined optimization objective.
However, they suffer from poor adaptivity to new environment, high development cost and prolonged development cycle.
Discrete system simulation method simulates real-world manufacturing environment by creating simulation models, from which general principles are difficult to find due to its experimental nature.
Rolling horizon rescheudling was original proposed by Nelson in 1977 and has received numerous research attentions and applications in recent years.
It divides the dynamic scheduling problem into continual scheduling sections and conducts online optimization for each section in order to find the individual optimal solution, which makes it applicable to complex dynamic manufacutring scheduling environments \citep{bierwirth19991, jian19972}.

Genetic algorithm, one of the metaheuristic algorithms, has witnessed many applications in dynamic scheduling researches due to its simple operations, higher efficiency, better robustness, superior adaptability and intriguing searching capability based on individual fitness. 
This paper proposes a hybrid algorithmic framework of multi-objective genetic algorithm and rolling horizon rescheduling to solve the real-world dynamic scheduling problem considering delayed raw material arrival, part maching time and assembly time.
It can also tackle the emergent insertion of parts and continuous arrival of scheduling jobs.
This framework can be used in other manufacturing scenarios as well.



















