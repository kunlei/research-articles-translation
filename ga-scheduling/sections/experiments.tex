\section{Computational Experiments}
We developed a prototype scheduling system based on the described improved genetic algorithm and validated its performance based on testing instances.
Since there exist no benchmark instances in the literature specifically designed for dynamic scheduling problems, we create testing instances based on the widely used FT20 problem and add dynamic data when dynamic scheduling is required.
These testing instances are then used to simulate the scheduling problems in dynamic manufacturing environments where unexpected events and urgent jobs arrival happen often.
In the experiment, minimization of mean flow time is used as the objective function in the case of unexpected events; a multi-objective of maximal makespan and total tardiness is used as the objective function in the case of urgent job insertion.

Parameters of the improved genetic algorithm are set as follows: population size $P = 200$; crossover probability $P_c = 0.8$; crossover times $n = 10$; mutation probability $P_m = 0.01$ and total number of iterations as 100.
Figure ? shows the initial scheduling plan for the FT20 instance with 20 jobs to be processed on 5 machines.
This initial plan represents the optimal solution for this problem and the maximum completion time is 1165.
Suppose the following stochastic events happpen on time 600: 1) processing time of job $(J_{10}, 1)$ on machine 2 is delayed for 30 units of time, due to machine breakdown, work delay or other reasons; 2) jobs $(J_1, 3)$ and $(J_{20}, 3)$ on machine 3 are exchanged due to worker absenteeism or vacation; 3) starting time of job $(J_8, 1)$ on machine 3 is delayed for 60 time units due to delayed arrival of raw material; 4) starting time of job $(J_11, 4)$ on machine 5 is delayed for 40 time units due to delayed assembly;
The maximal completion time is delayed to 1219 after the manual adjustments to accomodate the above stochastic events.
The improved genetic algorithm is applied at time 600 to reschedule refreshed jobs and figure ? shows the optimal schedule of this scenario and the maximal completion time is reduced to 1202.


To futher validate the performance of rolling horizon optimization, we assume the arrival of urgent jobs at time 600 in addition to the aforementioned stochastic events.
There are three jobs in the urgent arrival, namely, job 21, 22 and 23, and they are required to finish before given delivery date.
Table ? gives the processing machines, processing times, release times and delivery dates of the three jobs.
Note that delivery date is defined by multipling average processing time by 1.8 and adding the result to the rescheduling time point of 600.
In this scenario, the scheduled jobs need to be processed as sson as possible, whicle new urgent jobs must be finished before delivery date.
To this purpose, we need to solve the multi-objective dynamic scheduling problem considering both maximal job completion time and total tardiness. 
Using this multi-objective function, the improved genetic algorithm is used to reschedule all the jobs at time 600 and figure ? shows the optimal solution.
It can be seen from the figure that all the three newly inserted jobs are finished processing before time 900 and the maximal completion time is 1388.

The above rescheduling strategy can be used together with the improved genetic algorithm to continuously optimize all the jobs within a rolling horizon.
The computational results show that a complex dynamic scheduling problem can be naturally divided into multiple scheduling horizons and static scheduling algorithm can be used to optimize job schedules within each horizon, which improves the scheduling capability in adapting to dynamicly changing manufacturing environmnets and urgent events and jobs can be tackled promptly.
This strategy can be used in other dynamic scheduling problems as well.