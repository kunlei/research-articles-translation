\section{Dynamic Scheduling Strategy}
This paper proposes an improved genetic algorithm within the rolling horizon framework utilizing the hybrid cycled and event-driven rescheduling strategy.
The job rescheduling within a horizon is achived by the proposed genetic algorithm and the solution can be used for exection.
The following sections introduce the main components of the dynamic scheduling problems.

\subsection{Dynamic scheduling solution encoding scheme}
This paper uses a process-based encoding scheme in which all the processes of a job are indicated by the job number, and the $k$th occurance of the job indicates the $k$th process of the job.

\subsection{Dynamic scheduling solution decoding scheme} 
The scheduling solutino decoding scheme refers to the determination of the starting times of all process on all machines based on the current solution chromosome and process plans.
For a process $i$, given a starting time $t_i$ and processing time of $p_i$, its completion time can be computed as $(t_i + p_i)$.
The preceeding process of process $i$ is denoted by $JP[i]$ and the preceeding process of machine is $MP[i]$ if it exists.
In dynamic scheduling problems, the machining starting times of jobs that are being processed or un-processing, and not the first process, can be calculated as 
\begin{align}
	t_i = \text{max}(t_{JP[i]} + p_{JP[i]}, t_{MP[i]} + p_{MP[i]})
\end{align}
the starting times of the first process can be computed as
\begin{align}
	t_i = \text{max}(\text{max}(t_{JP[i]} + p_{JP[i]}), t_{MP[i]} + p_{MP[i]})
\end{align}

For available processes, the starting time $t_i$ of the first process can be computed by 
\begin{align}
	t_i = max(r_i, t_{MP[i]} + p_{MP[i00]})
\end{align}

This paper uses a decoding algorithm based on greedy insertion and can gurantee to produce a feasible schedule after solution decoding.

\subsection{Crossover and mutation operators}
There exist many crossver operators in the literature, including PPX, SPX, POX, among which POX is able to inherit promising characteristics of parent solutions and is depicted in figure ?.

The mutation operator randomly selects a gene and inserts it into another position in order to produce a small solution perturbation.


\subsection{Selection operartor}


\subsection{Improved genetic algorithm design}

